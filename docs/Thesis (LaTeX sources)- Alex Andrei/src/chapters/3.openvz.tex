\chapter{OpenVZ Commander Module} \label{chapter:openvz}
The \texttt{\project}application wishes to provide support for container 
virtualization solutions and operating system layer virtualization. \texttt{OpenVZ}
is currently an open-source container virtualization technology built for 
Linux. The virtual environment (known as \emph{container}) is essentialy a 
complete Linux file system tree mounted in the host file system and 
isolated using \texttt{chroot} command. This operation allows processes to run 
using a different root directory, which means it can not access files 
outside of specified directory tree. As a result, each container can act 
like a stand-alone system, with it's own:
\begin{itemize}
  \item users
  \item IP addresses
  \item files
  \item applications
\end{itemize}

Container virtualization implies the creation of the directory tree and 
configuration file. A large number of precreated \texttt{templates} are 
available on the OpenVZ download site\footnote{\url {http://download.
openvz.org/template/precreated/}} 
which can be easily imported using the OpenVZ API (\texttt{vzctl}). 
In regards to the aspect of creating new containers, \texttt{OpenVZ} is 
fairly easy to use, by comparison with the other important container virtualization 
technology, \texttt{lxc} (linux containers). However, \texttt{lxc} has the 
advantage of having native support in the Linux kernel. Instead, \texttt{OpenVZ} 
requires a new kernel, compiled to offer support for container virtualization. 
Some Linux distributions, like Debian or Fedora, make this operation easier for 
the user, by providing the compiled kernels in their repositories. The user needs 
only to install the new kernel and the API tools.

\fig[scale=0.7]{src/img/openvz.png}{img:arch}{OpenVZ Arhitecture}

However, most of the distributions (most notably, Ubuntu) do not offer such 
support and the user must manually download the kernel, compile it with 
container virtualization support and install it. This operation can be fairly 
complicated. Considering these aspects, the system used as a host for the 
containers, during testing was a \texttt{Fedora 14}.

\section{General Design} \label{sec:openvz-design}
This section describes the basic implementation of the \texttt{OpenVZ 
commander module} and its connection with the other modules of the 
application. A commander module is the application component 
that creates the virtual machine and it's hardware configuration and 
instantiates the proper modules to handle further configurations (networking, 
users, applications, etc ). This module is written in \texttt{Python} but 
also makes use of \texttt{bash scripts} for certain operations. 

A common interface ( \texttt{vmgCommanderBase}) is used for all the 
commander modules and contains the basic operations that need to be 
implemented by any commander. This application architecture facilitates 
the addition of new commander modules for new virtualization technologies 
without having to alter the existing modules. 
\\
\lstset{language=Python,caption=Commander module interface,
label=lst:if-commander}
\begin{lstlisting}
class CommanderBase:
	def startVM(self):
		pass
	def shutdownVM(self):
		pass
	def connectToVM(self):
		pass
	def disconnectFromVM(self):
		pass		
	def setupHardware(self):
		pass
	def setupPartitions(self):
		pass
	def setupOperatingSystem(self):
		pass
	def setupServices(self):
		pass
	def setupDeveloperTools(self):
		pass
	def setupGuiTools(self):
		pass	
	def getConfigInstance(self):
		return None
	def getInstallerInstance(self):
		return None
\end{lstlisting}

As seen above, the commander must provide methods for starting or 
shuting down the virtual machine, for hardware setup (including partitions 
and operating system, if necessary) and for applications setup (services, 
developer tools, gui tools).

Also, the commander is responsible 
for instantiating an \texttt{installer module} that will install the programs 
requested by the user and a \texttt{config module} that will perform 
additional configuration to the virtual machine (networking, users).

The \texttt{OpenVZ commander} overrides only a subset of these 
methods. For example, the \texttt{setupOperatingSystem} method 
does not have a meaning in the context of container virtualization. 
The following sections will describe in detail how the \texttt{OpenVZ 
module} implements each of the methods.

\subsection{Module Control Flow} \label{sec:openvz-flow}
The \texttt{OpenVZ commander module} is responsible for creating 
and configuring a new \texttt{OpenVZ container} using the settings 
specified in the configuration file. As shown in the previous section, the 
configuration file is parsed, and the subsequent data structure is serialized 
and dumped in a file. Also, the central module \texttt{vmgen} is responsible 
for instantiating the corresponding commander module (OpenVZ, lxc, 
VMware ).
\\
\lstset{language=Python,caption=Retrieve settings from the dump file,
label=lst:load-dump-file-2}
\begin{lstlisting}
class CommanderBase:
	def __init__(self, dumpFile):
		self.loadStruct(dumpFile)

	def loadStruct(self, dumpFile):
		with open(dumpFile, 'rb') as f:
			self.data = pickle.load(f)
	...

class CommanderOpenvz(CommanderBase):
	def __init__(self, dumpFile):
		CommanderBase.__init__(self, dumpFile)
	...
\end{lstlisting}

The first thing the commander does is to restore the data 
structure containing all the user-specified setting from the dump file and 
store it locally for further user. After retrieving the settings data structure, 
the commander will perform the basic setup in the following order:
\\
\lstset{language=Python,caption=Operations performed by the 
commander module,
label=lst:setup-order}
\begin{lstlisting}
	self.setupHardware()
	self.setupPartitions()
	self.setupOperatingSystem()

	self.startVM()
	self.connectToVM()

	self.config = self.getConfigInstance()
	self.config.setupConfig()
	self.root_passwd = self.config.getNewRootPasswd()

	self.installer = self.getInstallerInstance()
	self.setupServices()
	self.setupDeveloperTools()
	self.setupGuiTools()

	self.disconnectFromVM()

	self.shutdownVM()
\end{lstlisting}

The commander first creates the virtual machine and the \texttt{setup 
hardware} and then starts the virtual machine. Subsequent operations 
require the vritual machine running. Next, it will instantiate a 
\texttt{configuration module} and an \texttt{installer module} that will 
complete the configuration. After these operations are completed, the 
vritual machine is ready and is packed in order to be transferred to the user. 
The directory will be packed in an archive file and this will also be packed 
together with the container configuration file in the final archive which 
will be passed to the user.

\subsection{OpenVZ API} \label{sec:openvz-api}
The \texttt{OpenVZ technology} provides an extensive tool for container 
management: \texttt{vzctl}\cite{vzctl}. This facilitates the creation and removal, 
startup and shutdown, and the setting of a wide range of parameters for 
the container\cite{guide}. However, the \texttt{OpenVZ commander} uses only a 
subset of the options available, since many of them are used for extreme 
low-level configurations. For example, the option to set the maximum size 
of the TCP receive buffer ( --tcprcvbuf) is highly unlikely to appear in the 
user requests. Also, the inclusion of these options would have made the 
creation of a common interface for all the virtualization solutions extremely 
difficult.

The subset of options used by the commander is shown below:
\begin{itemize}
  \item vzctl \texttt{create} CTID --ostemplate name : creates a new container with the specified 
id, from a precreated template
  \item vzctl start | stop | destroy CTID : starts, stops or destroys the container
  \item vzctl set CTID --cpus num : number of CPUs available in the container
  \item vzctl set CTID --privvmpages value : controls the amount of memory available
  \item vzctl set CTID --diskspace value : controls the amount of harddisk available
  \item vzctl set CTID --netif_add ifname : adds a new virtual interface (veth)
\end{itemize}

Also, the module does not use some useful options that can be set using the \texttt{vzctl} 
API, like IP addresses, nameservers, hostname. These are set by the \texttt{configuration 
module} (configLinux.py)  in a later stage, using Linux commands instead. It is much easier 
to use a configuration module for Windows and Linux to handle all possible virtualization 
solutions, instead of creating configuration modules for each technology.

\subsection{Communication Interface} \label{sec:openvz-comm}
Each commander uses a \texttt{communicator module} to handle the communication with the 
virtual machine instance. A separate communicator is required for each virtualization solution 
and must offer a certain interface, defined by the \texttt{vmgCommunicatorBase} class.

\fig[scale=0.5]{src/img/communicators.png}{img:cmd-if}{Common commander interface}

This interface must provide functions for executing commands in the virtual machine 
(container), for copying and deleting files inside the virtual machine.

\fig[scale=0.4]{src/img/vm-communication.png}{img:comm}{Communication with the virtual machine}

Each communicator is instantiated with a set of communication parameters specific to each 
virtualization technology. For \texttt{OpenVZ}, these parameters are:
\begin{itemize}
  \item \texttt{vmx}: path to the VMware virtual machine used as a host for the container
  \item \texttt{host}: user and hostname for the VMware virtual machine
  \item \texttt{id}: the id of the container
\end{itemize}
The communicator will use \texttt{ssh} and the \texttt{vzctl} API to execute commands 
in the VMware host and \texttt{scp} for file transfers:
\\
\lstset{language=Python,caption=Communication with the container,
label=lst:container-comm}
\begin{lstlisting}
def runCommand(self, cmd):
	executeCommandSSH("vzctl enter " + self.id + " --exec " + cmd + ";logout")
	
def copyFileToVM(self, localPath, remotePath):
	executeCommand("scp " + key + " " + localPath + " " + self.host + ":$VZDIR/root/" + self.id + "/" + remotePath)
	
def deleteFileInGuest(self, remotePath):
	executeCommandSSH("rm -rf $VZDIR/root/" + self.id + "/" + remotePath)
\end{lstlisting}

The file transfers actually represent a simple copy operation to the container directory tree 
which is mounted in a specific location in the host file system. The location where the containter 
is mounted is given by the \texttt{\$VZDIR} variable, the \texttt{root/} subfolder and then the subfolder corresponding to the container, which has the name of the \texttt{container id}.

\section{Basic Configurations} \label{sec:openvz-basic-conf}

The settings for the hardware setup are listed in the \texttt{hardware} section of the 
configuration file:
\\
\lstset{language=Python,caption=Hardware section example,
label=lst:hardware}
\begin{lstlisting}
[hardware]
	vm_id = 123
	os = fedora-14-x86
	num_cpu = 1 
	ram = 1024
	[[hdds]]
		[[[hdd0]]]
		size = 1G
	[[eths]]
		[[[eth0]]]
		type = nat
		connected = 1
\end{lstlisting}

For the creation of the container, the module uses a \texttt{precreated template} in the form of a gzip 
archive, containing the directory tree. The name of the template is set using the \texttt{os} 
option. The module will use a \texttt{bash script} to make sure the template is present in the OpenVZ 
cache (/vz/template/cache). If the template is not found, it will try to retrieve it from the 
OpenVZ download page\footnote {\url {http://download.openvz.org/template/precreated}}.
\\
\lstset{language=bash,caption=Bash script to retrieve template,
label=lst:get-template}
\begin{lstlisting}
#!/bin/bash
...
get_template()
{
	pack=$1.tar.gz
	# possible download paths
	precreated="http://download.openvz.org/template/precreated/"
	beta=$precreated"beta/"
	contrib=$precreated"contrib/"
	unsupported=$precreated"unsupported/"
	# wget: set quite mode && output directory
	flags="-q -P /vz/template/cache"
	for link in $precreated $beta $contrib $unsupported
	do
		echo " * try $link$pack"
		wget $flags $link$pack
		if [ $? -eq 0 ]; then
			exit 0
		fi
	done
	echo "template not found"
	exit 1
}
...
\end{lstlisting}

For a complete view of the bash script used for template retrieval, please consult 
the \labelindexref{annex}{chapter:get-template}.

If the container was succesfully created, the setup continues by starting the 
container and setting the following options, using the \texttt{vzctl} API 
mentioned earlier:
\begin{itemize}
  \item number of CPUs available inside the container
  \item amount of memory available
  \item amount of harddisk available
\end{itemize}

The most interesting aspect of the hardware setup is the network interfaces setup. 
OpenVZ provides \texttt{veth}(Virtual Ethernet)\cite{veth} and \texttt{venet}(Virtual Network) 
devices for networking. The differences between the two are shown below
\footnote {\url {http://wiki.openvz.org/Differences_between_venet_and_veth}}:
\\
\\
\begin{tabular}{ | c | c | c | }
  \hline
  \texttt{Feature} & veth & venet \\ \hline
  \texttt{MAC address} & Yes & No \\
  \texttt{Broadcasts inside CT} & Yes & No \\
  \texttt{Traffic sniffing} & Yes & No \\
  \texttt{Network security} & Low\footnotemark[2] & High\footnotemark[3] \\
  \texttt{Can be used in bridges} & Yes & No \\
  \texttt{Performance} & Fast & Fastest \\
  \hline
\end{tabular}
\footnotetext[2] {Independent of host. Each CT must setup its own separate 
network security.}
\footnotetext[3] {Controlled by host.}
\\
\\
The module can setup up networking using any of these two, according to the 
option \texttt{type} specified for each interface. \texttt{Venet} interfaces only 
require an IP address. However, \texttt{veth} interfaces needs additional configuration. 
The commander uses \texttt{netif_add} options in the \texttt{vzctl} API to create 
an interface in the container, an interface in the host and, most importantly, a 
bridge between them. Furthermore, \texttt{IP forwarding} and \texttt{proxy_arp} 
must be enabled for both interfaces,  \texttt{IP addresses} must be assigned to 
each interface (must be in the same network) and a new default route added in 
the container. In order to make these persistent, a script including these operations 
is set to execute at container start-up\footnote{\url{http://www.linuxweblog.com/blogs/sandip/20080814/bridge-networking-on-openvz-containers-using-veth-devices}}.

\section{Additional Configurations} \label{sec:openvz-extra-conf}
Besides the hardware setup, other additional configurations are required for the 
container. These include:
\begin{itemize}
  \item \texttt{Users setup}: groups, users, passwords
  \item \texttt{Network setup}: IP addresses, nameservers, hostname
  \item \texttt{Firewall setup}: open ports, allowed programs, additonal firewall rules
  \item \texttt{Applications setup}: install programs
\end{itemize}

These configurations are not done explicitly by the \texttt{OpenVZ commander} module, 
but by additional components. The reason is that these additional configurations 
are not related to the virtualization solution. For example, network setup is similar 
on Linux virtual machines, even if they are OpenVZ container, lxc containers or 
VMware Linux guests. Also, the application installing does not depend on the 
virtualization technology, but on the operating system installed:
\begin{itemize}
  \item Windows virtual machines use \texttt{InstallerWindows}
  \item Linux virtual machines use \texttt{InstallerApt}/\texttt{InstallerYum} depending on the 
distribution (Debian/Fedora/Ubuntu)
\end{itemize}

However, the commander is responsible for instantiating these additional components 
with the correct parameters (configuration file settings, list of programs to be installed, 
communication module).

The following subsection will describe the \texttt{configuration module} for Linux. For the 
\texttt{Linux installer modules}, please consult \labelindexref{chapter}{chapter:linux-inst}.

\subsection*{Linux Configuration Module} \label{sec:linux-conf}
As with all the modules of the application, the Linux configuration module is 
required to implement a specific interface.
\\
\fig[scale=0.4]{src/img/configs.png}{img:conf-if}{Common configuration interface}

The \texttt{vmgConfigBase} class specifies that each configuration module must 
provide methods for system configuration, groups and user setup, network and 
firewall setup.
\\
\lstset{language=Python,caption=Configuration module interface,
label=lst:if-configure}
\begin{lstlisting}
	def setupConfig(self):
		self.setupSystem()
		self.setupGroups()
		self.setupUsers()
		self.setupNetwork()
		self.setupFirewall()

		self.applySettings()

	def setupSystem(self):
		pass
	def setupGroups(self):
		pass
	def setupUsers(self):
		pass
	def setupNetwork(self):
		pass
	def setupFirewall(self):
		pass
	def applySettings(self):
		pass
	def getNewRootPasswd():
		return None
\end{lstlisting}

The methods will be automatically called by the \texttt{setupConfig} 
in the correct order (for example, groups before users). The modifications made 
by all these methods will be applied to the virtual machine by the 
\texttt{applySettings} method. Furthermore, each configuration module must 
provide the new access credentials (\texttt{getNewRootPasswd}) for the virtual 
machine, in order to be used by the subsequent modules (the installer modules).

In detail, the Linux configuration module uses Linux specific commands to 
handle the necessary configurations. In some cases, multiple commands can be 
used to achieve the required result. However, the commands used were the ones 
who have a \emph{permanent} or \emph{non-interactiv behavior}. For example, multiple commands 
can be used to set the \texttt{hostname}:
\\
\lstset{language=bash,caption=Set hostname on Linux systems,label=lst:set-hostname}
\begin{lstlisting}
sysctl kernel.hostname=new_hostname
hostname new_hostname
edit /etc/sysconfig/network and modify hostname value
\end{lstlisting}

From the listed commands, the second does not modify the system permanently, 
while the third is harder to use non-interactively. So, the first command is 
preferable to be used.

For group and user setup, the \texttt{groupadd} and \texttt{useradd} commands 
are used, as they are both permanent and can be used non-interactively. For 
modifying the user passwor, the \texttt{passwd} command can be used, using a 
standard input redirect and the \texttt{echo} command.
\\
\lstset{language=Python,caption=Set user password non-interactively,
label=lst:user-pwd}
\begin{lstlisting}
	echo new_pwd | passwd --stdin user
\end{lstlisting}

In regards to the network setup, the \texttt{ip} command (iproute2 package) 
can be used for most of the settings (\emph{ip addr}, \emph{ip link}). However, in some 
cases file editing is also used - for example, nameserver setup requires the 
editing of file \emph{/etc/resolv.conf}.

For firewall setup, open ports and firewall rules can be both configured by 
using \texttt{iptables}.

Each of the previous configurations are added to a script file, which in the 
end will contain all the requested settings. To apply these settings, the 
script file is copied and executed inside the virtual machine.

This operation completes the system configuration and the virtual machine can 
subsequently be used for software configuration - application install.

\section{Testing Results} \label{sec:openvz-testing}
The commander module has been thoroughly tested to generate various containers 
for several Linux distributions. Since the time of creating a container is 
mostly spent on generating the directory tree, and OpenVZ offers the set of 
precreated templates for this, the time results vary around the 1-2 minutes margin.

The configuration module has been tested both as a stand-alone component and 
in connection to the commander module with good results. The settings are 
succesfuly applied to the virtual machine in under 1 minute.