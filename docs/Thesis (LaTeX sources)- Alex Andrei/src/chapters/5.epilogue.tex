\chapter{Conclusion} \label{chapter:epilogue}
Despite the limited amount of time, we managed to provide most of the 
features we intended for the application. The tool succesfully generates and 
configures virtual machines for several important virtualization technologies. It
is rather easy to use and logs details about each operation being performed.

However, there is always room for improvement, so there are a few aspects 
that can be improved and some features that can be added.

In order to improve the usability aspect, the application definitely would 
benefit from the use of a \emph{graphical user interface} which would make 
it easier for the user to generate the configuration file.

Also, the application requires a public framework which would enable users to access 
it, test it and provide valuable feedback.

Regarding the software configuration, some programs are not supported on some 
systems, due to repository issues. Also, a source installer module would greatly 
benefit \project, as it would allow it to install programs on any Linux distribution.

One of the best features of \project is the application architecture which makes it easy to 
extend. This means that new commander modules can be added fairly easy, in 
order to provide support for other technologies, like \texttt{Virtual Box} or \texttt{Kvm}. Also, 
new Linux installer modules can be added to offer support for other package 
managers, like Gentoo's emerge.

Another aspect that can be improved is the complexity of the system 
configuration. The application manages to support a large set of settings, 
which we felt were more likely to be requested by users. However, this is some 
room for expansions in this area, too. This can also be applied to the set of 
supported programs which can easily be extended.

Another nice feature would be the use of templates for the virtual machine 
configurations. This way, users that intend to use the virtual machine for 
development purposes could use a \emph{Development} template to configure a 
virtual machine with text editors, system libraries and other development tools 
installed. Users that are going to use the virtual machine for networking 
purposes could use a \emph{Networking} template to configure the system with a 
http server, nmap, wireshark or other networking tools. This kind of options 
will certainly speed up the tool, since the generation of the configuration 
file can be rather time consuming.

In retrospect, the work on this project has been both challenging and 
interesting. The application is a useful tool for various types of users and 
can be considered a good starting point for future development.