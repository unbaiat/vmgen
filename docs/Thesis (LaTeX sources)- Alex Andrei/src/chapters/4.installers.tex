\chapter{Linux Installers} \label{chapter:linux-inst}

The final stage of the machine configuration is the the installation 
of programs inside the virtual instance. These are a set of applications 
that are commonly used on these system (for example, development 
tools) and are likely to be requested by the users. The set was 
completed using our previous experience in using virtual machines 
for development purposes. The list of programs is defined in 
the configuration file:
\\
\lstset{language=Python,caption=User defined list of pograms,
label=lst:app-list}
\begin{lstlisting}
[devel]
	vim = 1
	valgrind = 1
	python = 1
	openjdk = 1
[services]
	mail_server = 1
	dns_server = 1
	sshd = 1
	git = 1
	mercurial = 1
\end{lstlisting}

The structure of the configuration file is flexible and the user can add 
various new programs, but \texttt{\project} will only consider a specific 
set of programs. Any other values added to the configuration file will 
be ignored.

The list of programs available for Linux is extensive and includes:
\begin{itemize}
  \item \texttt{Networking tools}: nmap, traceroute, netcat, etc
  \item \texttt{Development tools}: vim, emacs, valgrind, etc
  \item \texttt{Services}: httpd, dns server, sshd, git, etc
  \item \texttt{GUI tools}: web browsers, wireshark, etc
\end{itemize}

For the complete list of programs please consult the project's wiki page
\footnote{\url{http://ixlabs.cs.pub.ro/redmine/projects/vmgen/wiki/SupportedOptions}}.

The application provides installers for the distributions using \texttt{apt-get} or 
\texttt{yum} package manager.
\\
\fig[scale=0.4]{src/img/installers.png}{img:inst-arch}{Installer architecture}

Both use a common interface, extending the \texttt{InstallerBase} class and 
overriding it's \texttt{install} method.
\\
\lstset{language=Python,caption=Common installer interface,
label=lst:if-installer}
\begin{lstlisting}
class InstallerBase:
	def __init__(self, communicator):
		pass

	def install(self, program):
		pass
\end{lstlisting}

The architecture is flexible, allowing the addition of new installers for 
other package managers or a source installer that will compile and install 
the programs from their source code. 

Each installer uses a communicator passed by the commander module. 
The communicator is used by the installer to execute commands and 
copy files inside the virtual machine. This means that the installer does 
not need to distinguish between the virtualization solutions. The
transparency is ensured by the communicators.

\section{Apt-get Installer} \label{sec:apt-inst}
The module \texttt{InstallerApt} is designed to install programs for Linux 
distributions that use the \texttt{apt-get} package manager. It uses the 
\texttt{Debian/Ubuntu repository} to retrieve packages for each program and 
install it accordingly. Each program is identified in the repository by the name 
of the package. This means that the installer will have to keep a data structure 
that maps each program to a certain package name. This data structure is 
stored as a \texttt{Python dictionary} inside the \texttt{InstallerApt} module.
\\
\lstset{language=Python,caption=Apt package name mapping,
label=lst:apt-dict}
\begin{lstlisting}
packages = {
	"mozilla-firefox" : {
 		"package" : "iceweasel firefox" },
	"mozilla-thunderbird" : { 
		"package" : "icedove thunderbird" },
	"pidgin" : {
 		"package" : "pidgin" },
	"wireshark" : {
 		"package" : "wireshark" },
	...
}
\end{lstlisting}

As seen above, the package name can have multiple values for a certain 
program, as some packages change their name in time or differ between 
\texttt{Debian} and \texttt{Ubuntu}. For example, the \texttt{Mozilla 
Firefox} browser package is known as \texttt{iceweasel} on \texttt{Debian} 
and \texttt{firefox} on \texttt{Ubuntu}.

Some programs (for example, \texttt{eclipse} or \texttt{mpich2}) that can 
be found in the \texttt{Ubuntu} repository don't have a correspondent in 
the \texttt{Debian} repository. Also, the repository version of \texttt{eclipse} 
is old and the correct way to install it would be to do it manually instead of 
using the repository. 

As a result of these problems, a small set of programs are not available on 
some systems.

Regarding the module control flow, the \texttt{installer} will first check the 
list of programs received from the commander module and will only keep 
those that can be found in the \texttt{packages} mapping. Any other 
programs are not supported and considered invalid. Each programs will 
then be installed by executing the \texttt{apt-get} command inside the 
virtual machine using the communicator as a mediator.
\\
\lstset{language=Python,caption=Programs install,
label=lst:install}
\begin{lstlisting}
install_cmd = " /usr/bin/apt-get install -y -q "
class InstallerApt:
	def install(self, programs):
	...
	# Retrieve only the list of valid programs
	packs = [packages[p] for p in programs if p in packages]
	if packs:
		[self.comm.runCommand(install_cmd + p['package']) for p in packs]
\end{lstlisting}

The \texttt{apt-get} command is run in \texttt{quiet mode} (" -q ") since 
the output is not relevant. Also, the command will run 
\texttt{non-interactively} (" -y ") and will assume \texttt{yes} as answer to all 
prompts\cite{apt}.

Another problem encountered in the testing stages on some systems is the 
presence of an option in the \texttt{apt-get} sources file that enables the 
command to search the requested package on the installing CD-ROM. However, 
if this is not found, the command will block waiting for user actions, thus 
blocking the entire module. To fix these, a \texttt{sed} command will disable 
any such options by commenting the correspoding lines in the sources file.
\\
\lstset{language=Python,caption=Disable cdrom option,
label=lst:disable-cdrom}
\begin{lstlisting}
self.comm.runCommand("sed -i '/cdrom/s/^/# /' /etc/apt/sources.list")
\end{lstlisting}

\subsection *{Testing results}
This installer has been tested more as a stand-alone module, independently 
from the central module. This has the advantage of not wasting time on each 
test with the virtual machine creation and system configuration.

The tests included various sets of applications, including both small and large 
packages. The time result can vary significantly as a result, ranging from several 
seconds, for sets of small packages, to several minute for lage packages.

To install all the applications supported, the application required up to 10 
minutes. This figure can also vary, depending on the Internet connectivity and 
the host system. The time can not be improved by the application as it is only 
related to package manager operations, the retrieval of packages and the actual 
install.

The module has also been tested, with good results, in the context of the 
entire application, to check the connectivity between this module and the 
corresponding commander module.

The module has been tested on Debian 6 and Ubuntu 10.10 distributions. 
For some older distributions, the installer does not guarantee the succesful 
install of all the programs specified. This can easily be fixed with an
extended testing stage.

\section{Yum Installer} \label{sec:yum-inst}
The module \texttt{InstallerYum} is designed to install programs for Fedora
distributions that use the \texttt{yum} package manager. This is similar to 
the \texttt{apt-get} repository mentioned in the previous 
\labelindexref{section}{sec:apt-inst}. Each program is identified by a package 
name. In addition, a set of programs (for example, development tools) can be 
installed together as a group. In this case, the identifier is the name of the 
group\cite{yum}.

Considering the similarities, the structure of the \texttt{Yum installer} will 
basically be the same, mapping each program to a package name. However, 
each install entry will also require a \texttt{type} value to distinguish the 
groups of packages from simple packages. Another option \texttt{local} 
is supported to enable the install of programs from local \texttt{rpm} files, 
but still using the \texttt{yum} package manager.
\\
\lstset{language=Python,caption=Yum package name mapping,
label=lst:yum-dict}
\begin{lstlisting}
packages = {
	"mozilla-firefox" : {
		"type" : "simple",
		"package" : "firefox" },
	"mozilla-thunderbird" : {
		"type" : "simple",
		"package" : "thunderbird" },
	"build-utils" : {
		"type" : "group",
		"package" : "\"Development Tools\" \"Development Libraries\"" },
	...
}
\end{lstlisting}

Some programs are not present by default in the \texttt{yum repository}. 
These programs require the addition of a specific \texttt{repo} file to the 
\texttt{yum} configuration files (the default location is \texttt{/etc/yum/repos.d}). 
The path to such \texttt{repo} files must be provided, as the installer needs to 
copy it inside the virtual machine. These programs can be identified in the 
packages dictionary by their type, which is set to \texttt{repo}.

Almost the entire set of available programs for the Linux guests is available in the 
\texttt{yum} repository, the most important exception being the \texttt{Chrome} 
web browser. The best way to install it is to use \texttt{Google's own yum 
repository}, using a \texttt{repo} file. Also, \texttt{dynamips} is not supported 
by the yum repository, but it can be installed by using a \texttt{rpm} file.

The control flow of this installer is similar to the \texttt{Apt installer}. This means 
that the module will first check the programs requested by the user and keep only 
the ones that are supported (the program name is a key in the \texttt{packages} 
dictionary. Afterwords, each program will be installed according to it's type.
\\
\lstset{language=Python,caption=Install commands,
label=lst:inst-yum-cmd}
\begin{lstlisting}
local_cmd = " yum -y -d 0 -e 0 localinstall --nogpgcheck "
simple_cmd = " yum -y -d 0 -e 0 install "
group_cmd = " yum -y -d 0 -e 0 groupinstall "
\end{lstlisting}

The \texttt{yum} command is run non-iteractively (" -y") and assumes \texttt{yes} 
to all prompts. Also, the output of the command is irrelevant and is therefore 
suppressed (the debug level and error level are set to 0).
\\
\lstset{language=Python,caption=Install commands,
label=lst:inst-cmd}
\begin{lstlisting}
if p['type'] == 'simple':
	self.comm.runCommand(simple_cmd + p['package'])
if p['type'] == 'group':
	self.comm.runCommand(group_cmd + p['package'])
if p['type'] == 'repo':
	# copy repo file in container
	self.comm.copyFileToVM(p['repo'], p['package'] + ".repo")
	# install package
	self.comm.runCommand(simple_cmd + p['package'])
if p['type'] == 'local':
	# copy rpm to root
	self.comm.copyFileToVM(p['rpm'], ".")
	# execute
	self.comm.runCommand(local_cmd + p['package'])
	# remove rpm
	self.comm.runCommand("rm -rf *.rpm")
\end{lstlisting}

The install command will differ for simple packages, group packages and local 
packages. In addition, the local installation requires the \texttt{rpm} file to 
be copied inside the virtual machine, before running the \texttt{yum} 
command with a specific flag. Also, the \texttt{repo} type programs 
need the configuration file copied to the guest, but are installed as simple 
packages afterwords.

\subsection *{Testing results}
Similar to the \texttt{Apt} installer, this component has also been tested more 
from a stand-alone perspective.

The tests included various sets of applications, including both small and large 
packages. To install all the applications supported, the application required 
up to 12 minutes. This figure can vary, depending on the Internet 
connectivity and the host system.

The connectivity with the commander module has also been tested, by running 
the entire application with all the modules included.

The module has been tested on Fedora 14 distribution. However, tests for older 
distributions are required in order to fix any problems that may arise because 
of the changes in the repository.