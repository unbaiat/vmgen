\chapter{Introduction} \label{chapter:intro}
Virtualization is certainly one of the most interesting topics in today's IT industry and 
academia. It is becomming more widely used by companies in various fields as 
it saves money, time and energy\cite{hess}. It is also an interesting topic of research 
and can be extremely useful for education purposes. 

To give a short definition, virtualization implies the creation of a virtual 
version of an operating system\cite{virt}. Another way of puting it, it enables the existence of 
multiple operating systems on a single computer, through the creation of 
\emph{virtual machines}. A virtual machine creates an \emph{abstraction} 
of the physical resources into virtual resources.

There are several different types of virtualization, depending on the level 
of \emph{abstraction}\cite{osc}. Some of these are:
\begin{itemize}
  \item \emph{full virtualization} - complete simulation of the hardware; operating 
systems run unmodified
  \item \emph{partial virtualization} - partial simulation; applications may require 
some modifications
  \item \emph{paravirtualization} - hardware is not simulated; virtual environments 
run on the same system as the host, but are isolated
\end{itemize}

They all have their advantages and disadvantage, some more than others, but each category includes 
virtualization solutions that are widely used nowadays. In this aspect, it is 
worth mentioning \texttt{VMware}, \texttt{Virtual Box}, \texttt{Kvm} (full virtualization), 
\texttt{OpenVZ}, \texttt{lxc} (operating system-level virtualization) and \texttt{Xen} (hypervisor).

Virtualization is an extremely useful tool that can be used in multiple 
scenarios. Since in enables the existence of multiple operating systems on a 
single machine, it takes away the necessity of running multiple machines, each 
with it's own operating systems. Also, because the operating systems are 
isolated, if one of them crashes, the others will not suffer any damage. This 
enables a virtual machine to act as a \emph{sandbox}, useful for testing 
and development purposes.

However useful they are, the process of generating and configuring a virtual 
machine is a time consuming operation. Installing an operating system, 
configuring a system and installing a set of applications takes a lot of time as 
they are operations done by the user manually. This is extremely slow compared 
to an automated process.

This paper introduces a tool called \texttt{\project} which tries to solve these issues by 
entirely automating these operations, thus reducing the total time spent. It's 
purpose is also to offer a common interface for the creation and generation of 
virtual machines, using several virtualization technologies. This is not easy to 
accomplish, since virtualization technologies do not tend to offer the same 
features as others. Each has it's own advantages and issues, so the task of 
finding the common aspects is rather challenging.

\texttt{Vmgen} also tries to simplify the entire process, thus enabling non-technical 
users to benefit from the advantages of virtualization without having to go 
through all the troubles of setting up a new operating system, a process which 
can get tedious sometimes.

Currently, the tool offers support for both full virtualization - \texttt{VMware} - and 
operating system-level virtualization - \texttt{OpenVZ} and \texttt{lxc}. It enables the creation 
and system configuration of Linux and Windows hosts. It also provides software 
configuration (programs and tools installed) for Windows and some important 
Linux distributions - Debian, Fedora, Ubuntu.

For the rest of the work, I will describe some of the components and modules 
of the applications, detail certain aspects of the implementation and offer 
some suggestions for the tool's future development.