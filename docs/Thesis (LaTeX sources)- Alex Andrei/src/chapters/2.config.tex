\chapter{Configuration File} \label{chapter:config}
The \texttt{\project}application requires the use of a configuration file at 
startup. This must be provided by the user and must contain the settings 
for the new virtual machine. The only setting not found in this file is the 
virtualizaton technology - that detail will be specified directly to the central 
module. The configuration file is the basis for all subsequent configurations.

The other option considered for providing the settings was an interactive 
prompt in which the user would specify the settings guided by the application. 
However, the use of a configuration file proved to be much more faster to use, 
although a little bit more difficult. Also, if a graphical interface would be 
created for the application, it would be much easier to integrate it and would 
not imply any modifications to the existing source code.

The following image shows the configuration file's role in the application's 
flow.
\\
\fig[scale=0.35]{src/img/application-flow.png}{img:flow}{Application flow}

\section{Format} \label{sec:config-format}
In order to make it easier to create, edit or interpret in later stages, the 
configuration file uses the \texttt{INI} file format to organize the data
\footnote{\url{http://en.wikipedia.org/wiki/INI_file}}. Another option 
considered was the use of the \texttt{XML} file format. This format uses the 
same basic representation (key, value pairs) and supports many of the same 
features as the \texttt{INI} format. Also, Python standard library supports 
both file formats (\texttt{XML} parsers and \texttt{INI} parsers). However, 
the \texttt{XML} syntax is much more complicated and dense. A configuration 
file in the \texttt{XML} file format can be \texttt{2-3} times larger than the 
equivalent in the \texttt{INI} file format. For these reasons, the \texttt{INI} 
format was preferred.

In the \texttt{INI} file format, each basic setting has a \texttt{name} 
and a \texttt{value}.
\\
\lstset{language=Python,caption=Basic configuration setting,
label=lst:conf-settings}
\begin{lstlisting}
	ip=192.168.1.2
	gateway=192.168.1.0
	nameserver=192.168.1.1
\end{lstlisting}

Related settings are grouped in \texttt{sections}. The major sections are:
\begin{itemize}
  \item hardware
  \item network
  \item users
  \item config
  \item devel
  \item services
  \item gui
\end{itemize}

These sections may contain nested sections - subsections, forming a hierarchy. A new 
subsection is marked by an extra pair of sorrounding brackets ("[...]"). Each major 
section will be presented in detail in \labelindexref{section}{sec:conf-features}. For more 
details about the content of the configuration file, please consult the project's wiki pages
\footnote{\url{http://ixlabs.cs.pub.ro/redmine/projects/vmgen/wiki/SupportedOptions}}.
\\
\lstset{language=Python,caption=Nested sections,
label=lst:subsections}
\begin{lstlisting}
[hardware]
	...
	[[hdds]]
		...
		[[[hdd0]]]
			...
			[[[[partitions]]]]
				...
\end{lstlisting}

The syntax is flexible, allowing the following features:
\begin{itemize}
  \item empty values
  \item white spaces
  \item empty lines
  \item list values
  \item multiple line values
  \item comments. 
\end{itemize}.
However, options outside the major sections will not be taken into account. 
Also, the user must take into account the fact that one section(major section 
or subsection) lasts until a token marking another section is found ("["). 
Identation is not required, but is highly recommended as it makes the 
configuration file much easier to interpret visually.

\section{Supported Features} \label{sec:conf-features}
As mentioned earlier, related settings are grouped in \emph{major} sections. The 
\emph{hardware} section contains the settings used for the actual creation 
of the virtual machine. It must contain a virtual machine identifier (name 
for VMware or lxc/numeric ID for OpenVZ) and an operating system identifier. 
Also, it can specify the amount of memory available to the virtual machine or 
the number of CPUs.

Furthermore, this section specifies the disk structure, by setting the hard 
drives, their parameters and the partitions. The cd drives can also be included 
in this section, as well as the network interfaces.

The rest of the sections include options used in the subsequent system 
configuration (networking, users) or software configuration (applications).

The \texttt{network} section contains the network parameters for each 
interface: IP address, gateway (on Windows), network mask or global 
parameteres: nameservers, gateway (on Linux). This section also includes 
firewall settings, by including a subsection that specifies the open ports and 
allowed applications. Also, for Linux, the section can include a subsection 
where the user can submit custom firewall rules (iptables). Besides these 
system settings, the section also includes a set of programs which can be 
categorised as networking tools (nmap, tcpdump, etc). These can be 
installed in the virtual machine.

The \texttt{users} section contains a list of custom groups and a list of new 
users with the corresponding parameters: password, group(s), home directory 
(Linux).

The \texttt{config} section contains some additional options for system 
configuration: hostname, root password, additional repositories (some Linux 
distributions).

The remaining three sections contain lists of programs which are to be 
installed in the virtual machine. These programs are grouped in three major 
categories:
\begin{itemize}
  \item \texttt{devel} - development tools
  \item \texttt{services} - services to be configured
  \item \texttt{gui} - tools with graphical interfaces
\end{itemize}

The \texttt{development tools} category includes:
\begin{itemize}
  \item text editors and IDEs - Vim, Emacs, Eclipse
  \item programming languages - Python, Php, Tcl
  \item debugging tools - valgrind
  \item support libraries - build-utils, kernel-devel, mpich2, openmp, openjdk
\end{itemize}

The \texttt{services} category includes:
\begin{itemize}
  \item servers: http, mail, dns, ...
  \item subversion systems: git, svn, mercurial
  \item network file systems: NFS, LustreFS
  \item additional services: ntp, squid proxy, bittorent
\end{itemize}

The \texttt{gui} section includes:
\begin{itemize}
  \item web browsers: Mozilla Firefox, Google chrome
  \item mail clients: Mozilla Thunderbird
  \item network analyzers: Wireshark
\end{itemize}

For more details on the supported configuration file options, please consult 
the project's wiki pages\footnote{\url
{http://ixlabs.cs.pub.ro/redmine/projects/vmgen/wiki/SupportedOptions}}.

\section{Editing and Parsing} \label{sec:conf-parsing}
The python standard library includes a module \texttt{ConfigParser} that can 
be used to parse configuration files that have a structure similar to INI files. 
However, the major disadvantage of this solution is that it does not support 
nested sections.

The structure of the configuration file without the use of nested sections 
can get extremely complicated. As a result, the application uses for parsing the 
\texttt{ConfigObj}
library which also adds some useful features, while maintaing the simplicity of use\cite{conf}.

The parsing module \texttt{vmgParser} just passes the 
name of the configuration file to \texttt{ConfigObj} which builds a data 
structure recursively from the content of the configuration file. Because of 
the way the options are specified ( list of pairs key=value ), an associative 
array must be used. This type of data structure is known in \texttt{Python} 
as a dictionary and contains a list of values indexed by keys. For this 
application, both the keys and the values will be represented as 
\emph{strings}. The object that represents the entire data strucure is, 
therefore, a dictionary which contains a list of dictionaries indexed by the 
name of the major sections. Each nested dictionary will also contain a list 
of dictionaries corresponding to it's nested sections.

To simplify the access to specific data in this structure, the application uses 
a wrapper class - \texttt{vmgSection} - which stores the information in a 
section (the corresponding dictionary) and provides an interface for handling 
basic operations.

The data structure built from the information in the configuration file is then 
serialized to be passed to subsequent modules. This means that the structure 
is converted to a format and stored in a file that can be later used to restore 
the original structure. This enables multiple modules to use the data structure 
without requiring parsing and building the structure each time. Also, the 
Python standard library supports object serialization through the use of the 
\texttt{pickle} library. The operations of creating the object serialization and 
writing it to a file (a process called \emph{dumping}), as well as restoring 
it is extremely fast and easy to use.
\\
\lstset{language=Python,caption=Serialized representation of a configuration
file,label=lst:dump-file}
\begin{lstlisting}
(ivmgStruct
vmgStruct
p0
(dp1
S'data'
p2
(dp3
S'network'
p4
(ivmgStruct
vmgSection
p5
(dp6
...
\end{lstlisting}

After the parsing stage, the \emph{dump file} is used by the 
\emph{commander} modules to restore the content of the configuration file 
and use it for the creation and subsequent configuration of the virtual 
machines.
\\
\lstset{language=Python,caption=Restoring the data structure,
label=lst:load-dump-file}
\begin{lstlisting}
class CommanderBase:
	def __init__(self, dumpFile):
		self.loadStruct(dumpFile)

	def loadStruct(self, dumpFile):
		with open(dumpFile, 'rb') as f:
			self.data = pickle.load(f)
\end{lstlisting}

Another important aspect is the editing of the configuration file. The 
structure is designed as to allow easy interpretation and editing of complex 
configurations. As a result, the configuration file is fairly easy to modify 
manually, by the user. However, it is rather difficult to make any 
modifications from another application. Doing this would require a re-parsing 
and familiarity with the data structure.

The \texttt{ConfigObj} library provides an interface for configuration file 
edition. However, \texttt{\project} modules wrap the library's features, 
masking it outside the application. For this reason, the editing feature of 
the library can not be directly used.

To solve this, another module was created in order to wrap around the editing 
feature of the library. It is easy to use by users, or by application, having 
an intuitive usage (it is similar to the use of \emph{sysctl} command in Linux 
systems).
\\
\lstset{language=Python,caption=Editing the configuration file,
label=lst:edit-conf}
\begin{lstlisting}
	vmgCtl.py network.eths.eth0.address=192.168.1.23
\end{lstlisting}

The above command accesses the \emph{network} section and the network 
interfaces subsection to set the IP address of interface eth0.

Similar commands can be used to alter or add any value in the configuration file, in 
any section or subsection.

One important notice is that the parsing modules of the application \emph{do not 
check} the validity of the data inside the configuration file. This means that 
user can set any options to invalid values without this being detected in the 
parsing stage. This kind of checking is rather complex for this stage and the
only use would be to fix user errors. Instead, the application will assume the data 
is correct. Of course, the actual error will be detected later on, by the 
appropriate module (commander, configuration module or installer).

Also, the addition of unrelated settings in the configuration file will not 
interfere with the operations of the other modules. For example, the user can 
add an option, specifying that he wishes a certain unsupported application 
installed.
\\
\lstset{language=Python,caption=Editing the configuration file,
label=lst:unsupported-setting}
\begin{lstlisting}
[gui]
	...
	visual_studio=1
	...
\end{lstlisting}
However, this option, although present in the data structure used by all the 
modules, will be ignored by the corresponding module (the installer module).

The conclusion is that the configuration file's syntax and content is 
highly flexible, but the options actually used by the modules of the 
application are restricted to a specific subset.

For more details about the structure and content of configuration files, please 
consult the various examples included in the \labelindexref{annex}{chapter:conf-samples}.