\chapter{Conclusion}
\label{chapter:conclusions}
So far, the application's core has been developed and tested. All the
components were individually tested, and then they were integrated and the
whole system was tested. I ran the full process for creating a VMware virtual
machine, with Windows XP installed, and 4 applications, and it took about 5
minutes. For the lxc container generation, I tested the creation of Debian and
Fedora containers, but without making the system configurations and installing
any applications.

The application is usable, but it offers a limited number of features: only a
small number of OS are available, a small number of applications are
installable, the user interaction is not very friendly. We focused on designing
the application and implementing it in a modular way, to make it easy to extend
it later and we didn't have the necessary time to polish the application and
make it fully usable. The project should be continued and the application
should be released. It will be easy to extend it with new features, because of
its structure. From my point of view, the application will be useful when it
will be finished and it will be able to handle a wide range of configurations
requests, allowing for a easier user interaction. Along with some caching
mechanisms to speed up the generation process, the application will allow the
users to obtain the needed virtual machines, spending much less time. They
don't need to interact with the process, and \project configures the machines
more quickly, because it can reuse some already generated pieces, and has no
downtime between the operations (the next operation starts exactly after the
current one is finished).
