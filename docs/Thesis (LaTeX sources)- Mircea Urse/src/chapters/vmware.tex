\chapter{VMware}
\label{chapter:vmware}

\section{Hardware Generation}
\label{sec:vmware-hw}

A VMware virtual machine has 2 main components: a machine description file
(\textbf{vmx}) and one or more virtual disks (\textbf{vmdk}). Each of these
must be generated and configured.

\subsection{The Machine Description File}
\label{sub-sec:vmx}
The machine description file contains the virtual machine name, information
about the OS and the hardware components of the machine (processors, memory,
network adapters, hard-disks, cd drives etc.). It is a list of property-value
associations, stored in plain text format. To generate the description file
for the new machine, the information from the hardware section in the config
file is used. The currently supported options in the config file are the OS
type, the name of the machine, the number of processors, the amount of physical
memory (RAM), the hard-disks, the cd drives, and the network adapters
parameters. For the hard-disks and cd drives, the file path, the controller
type (\textbf{ide}, \textbf{lsilogic} etc.) and the position on the controller
(\textbf{ide0:1}, \textbf{scsi0:5}) can be specified. For a network adapter,
the user can provide the type of connection to the host (\textbf{nat},
\textbf{bridge}, \textbf{host-only}), the hardware (MAC) address, and if the
adapter is powered on when the machine boots up. Only some basic options must
be specified in the \textbf{vmx} file. When the machine is powered on, the file
is updated and some additional options are automatically added (like hardware
addresses for the network cards if none specified, the PCI slot numbers etc.).

VMware does not provide an official list of options, but some extensive
unofficial references can be found online at \cite{vmx-sanbarrow}. A nice tool
I found when searching for the available options for the \textbf{vmx} file is
an online application\footnote{\url{http://www.easyvmx.com/}} which is able to
generate customized \textbf{vmx} files. The user can select the desired options
in a visual form, and then he is returned a generated \textbf{vmx} file.

A sample machine description (vmx) file, generated (by \project), is provided
in \labelindexref{Listing}{lst:sample-vmx}. For example, the \textbf{scsi0.*}
options specify the HDD controller's and the drive's parameters. In this case,
the controller is \textbf{SCSI}. For an \textbf{IDE} controller, the syntax is
similar to the one used for the cd-drives. The disks can be specified using
\textbf{scsiX:Y} or \textbf{ideX:Y} fields (where X is the SCSI/IDE
controller's number and Y is the controller's port index).

\lstset{caption=vmx file sample,label=lst:sample-vmx}
\begin{lstlisting}
#!/usr/bin/vmware
config.version = "8"
virtualHW.version = "7"
guestOS = "debian5-64"

numvcpus = "2"
memsize = "256"
displayName = "First GenVM"

# hard-disk
scsi0.present = "TRUE"
scsi0.virtualDev = "lsilogic"
scsi0:0.present = "TRUE"
scsi0:0.fileName = "first-disk.vmdk"

# cd-rom
ide1:0.present = "TRUE"
ide1:0.deviceType = "cdrom-image"
ide1:0.fileName = "debian-6.0.0-amd64-CD-1.iso"

# ethernet
ethernet0.present = "TRUE"
ethernet0.startConnected = "TRUE"
ethernet0.connectionType = "nat"
\end{lstlisting}

\subsection{The Virtual Disks}
\label{sub-sec:vmdk}
A virtual disk is a file that can be created with an utility provided by VMware,
\textbf{vmware-vdiskmanager}. It can be a single file, or split in multiple 2GB
size files. It is basically, like a physical disk. The creation utility
requires the size of the disk and the controller type (\textbf{ide},
\textbf{lsilogic} etc.). To create the desired virtual disks, the parameters of
each disk is retrieved from the config file and passed to the
\textbf{vmware-vdiskmanager} utility. Usage examples and the available options
can be found out by running the \textbf{vmware-vdiskmanager} without any
arguments. Example commands to create a SCSI and an IDE virtual disks are shown in \labelindexref{Listing}{lst:vmware-vdisk}.

\lstset{caption=vmdk creation commands,label=lst:vmware-vdisk}
\begin{lstlisting}
vmware-vdiskmanager.exe -c -s 800MB -a lsilogic -t 0 scsi-disk.vmdk
vmware-vdiskmanager.exe -c -s 7GB -a ide -t 0 ide-disk.vmdk
\end{lstlisting}


\section{Operating System Installation}
\label{sec:vmware-os}
One of the most important step in the virtual machine generation is the OS
installation. VMware offers an API to interact with the virtual machine, but
only after the OS is installed on it.  The machine can be controlled using the
VMware's \textbf{vmrun} command-line utility, or using the \textbf{pyvix}
framework. The \textbf{pyvix} framework is a wrapper written in Python over
\textbf{VIX}, the official API provided by VMware for controlling the virtual
machines programmatically. The method used by the \textbf{CommunicatorVmware} is
the \textbf{vmrun} one.

\subsection{Unattended OS Installation Solutions}
\label{sub-sec:vmware-unattended-solutions}
The first thing I tried to do was to find a way to install the OS unattended.
For Linux\footnote{\url{http://ixlabs.cs.pub.ro/redmine/projects/vmgen/wiki/AutoLinux}},
it is possible to provide a file containing the answers to the questions shown
during the OS install process. However, I found out that there are different
ways for different distributions: \textbf{preseed files} for Debian and Ubuntu,
and \textbf{Kickstart} for RedHat and Fedora. I then found out about
\textbf{fai}, an application which should be able to install any
distribution unattended, but I couldn't get it to work.

For Windows XP\footnote{\url{http://ixlabs.cs.pub.ro/redmine/projects/vmgen/wiki/AutoXP}},
I found a way to create a simple text file containing the information needed
during the setup process. This method didn't work in Windows 7\footnote{\url{http://ixlabs.cs.pub.ro/redmine/projects/vmgen/wiki/Auto7}}, for which I
found a large application which was able to generate an XML file containing the
answers.

\subsection{The Setup Used}
\label{sub-sec:vmware-setup}
Because there were different ways to install different systems, and because a
system installation is a time consuming operation, which changes very little
from one installation to another, I decided to take a different
approach\footnote{\url{http://ixlabs.cs.pub.ro/redmine/projects/vmgen/wiki/UnattendedInstall}}.
I manually installed some basic virtual machines with some of the OS supported
by \project and I stored their disks in a folder (their \textbf{vmx} files are
not necessary, because new ones will be created). The machines have only the OS
installed, along with \textit{VMware tools}, for easier interaction afterwards.
When the user requests a new machine, the corresponding base disk is selected
and the system partition is cloned from it onto the new disk.

An auxiliary machine is needed to partition the newly created disks and to
clone a base installation onto it. I configured an additional virtual machine
(\textbf{VMaster}), with a Debian system, with no GUI, and which was accessible
through SSH using public key authentication.

\subsection{Setting Up The Partitions}
\label{sub-sec:vmware-partitions-setup}
After all the disks of the new machines are created, they are attached to the
\textbf{VMaster}, along with the base disk corresponding to the desired OS. To
attach the disks to \textbf{VMaster}, the \textbf{vmx} file of the
\textbf{VMaster} is duplicated, and the necessary lines are added to the copy.
The duplication is necessary, to avoid looking for the new options
automatically added in the original file after the machine is started, in order
to remove the disks after the job is finished. By cloning, the copy is simply
discarded at the end of the process. The base disk is selected using a
dictionary stored inside the \textbf{CommanderVmware} module. The dictionary is
indexed by the OS identifier. The dictionary is shown in
\labelindexref{Listing}{lst:basedisk-dict} (only a limited number of systems is
supported at the moment; more will be added after the application is finished
and tested). Besides the name of the base disks (the path is a default one for
all of the disks), the dictionary has entries for other parameters of the
disks: the \textbf{MBR} type (\textbf{grub2}, \textbf{grub}, \textbf{win}), the
file system type of the system partition, the disk controller type.

\lstset{caption=sample OS based dictionary,label=lst:basedisk-dict}
\begin{lstlisting}
base_disks = {
	"debian5-64":{ 
		"name":"debian-64.vmdk",
		"type":"lsilogic",
		"mbr":"grub2",
		"fs":"ext4"},
	"ubuntu-64":{
		"name":"ubuntu-64.vmdk",
		"type":"lsilogic",
		"mbr":"grub2",
		"fs":"ext4"},
	"windows7-64":{
		"name":"Win7-64.vmdk",
		"type":"lsisas1068",
		"mbr":"win",
		"fs":"ntfs"},
	"winxppro":{
		"name":"WinXP.vmdk",
		"type":"ide",
		"mbr":"win",
		"fs":"ntfs"}
	}
\end{lstlisting}

After the \textbf{VMaster} is started, the application needs to wait for the OS
to boot up. This is accomplished using pyvix to wait for the VMware tools to
load inside the guest. The methods that implement this functionality are shown
in \labelindexref{Listing}{lst:pyvix-wait-tools}. \textbf{VMaster} needs to
have VMware tools installed, in order to use this feature. A thread waiting for
the tools to be ready is created and started. The thread only exits if the
tools are ready, or when it is killed by another one. To allow the application
to continue even if the thread did not return, a timeout is set. When the
timeout expires, an error code is returned.

\lstset{language=Python,caption=wait for VM to boot up,label=lst:pyvix-wait-tools}
\begin{lstlisting}
def connect_to_vm(vmx_path):
	try:
		# try defaults
		host = pyvix.vix.Host()
	except pyvix.vix.VIXException:
		print "error host"
	try:
		vm = host.openVM(vmx_path)
	except pyvix.vix.VIXException:
		print "error openVM"
	return (host, vm)

def _wait_for_tools(vm):
	try:
		vm.waitForToolsInGuest()
	except pyvix.vix.VIXException:
		pass

def wait_for_tools_with_timeout(vm, timeout):
	tools_thd = Thread(target = _wait_for_tools, args=(vm,))
	tools_thd.start()
	# normally the thread will end before the timeout expires, so a high timeout
	tools_thd.join(timeout)

	if not tools_thd.isAlive():
		return True
	return False
\end{lstlisting}

After the machine is started, the partition tables on each disk are created
according to the provided setup in the config file (in the partitions
sub-section). The \textbf{parted} utility is used to create the partition
tables. A problem I encountered was that the type for the \textbf{Windows}
partitions was reported correctly by \textbf{parted}, but it was reported as
\textit{Linux partition} in \textbf{fdisk} (which was the real type), and a
Windows system running on that partition wasn't able to boot.  The fix for this
problem was to use the non-interactive version of \textbf{fdisk},
\cmd{sfdisk}, to change the partition type to \textbf{Windows type}. To
create the file system on the partitions, the \textbf{mkfs*} utilities were
used. The dictionary used to select the specific utility for a partition is
displayed in \labelindexref{Listing}{lst:mkfs-dict}.

\lstset{caption=utilities to create file systems,label=lst:mkfs-dict}
\begin{lstlisting}
mkfs = { 
	"ntfs":{
		"cmd":"mkfs.ntfs", 
		"id":"7"},
	"ext2":{
		"cmd":"mkfs.ext2",
		"id":"83"},
	"ext3":{
		"cmd":"mkfs.ext3",
		"id":"83"},
	"ext4":{
		"cmd":"mkfs.ext4",
		"id":"83"},
	"swap":{
		"cmd":"mkswap",
		"id":"82"}
}
\end{lstlisting}

\subsection{Setting Up The OS}
\label{sub-sec:vmware-os-setup}
After the partitions are created, the system partition from the base disk is
cloned onto the partition designed as the new primary (the first partition of
the first new disk). For a \textbf{NTFS} system partition, the
\textbf{ntfsclone}, along with the \textbf{ntfsresize} utilities were used,
whereas for the \textbf{ext*} partitions, a simple \textbf{cp -ax} is used to
copy the files from one partition to the other. An alternative for the
\textbf{ext*} partitions is to use \textbf{dd}, but it takes more time to copy
(because it must copy all the blocks of the partition, not only the used ones).

Then, the MBR of the new system disk is updated, to make it bootable. For the
base disks having \textbf{grub} or a \textbf{Windows boot loader}, it is
sufficient to copy the first 446 bytes (out of 512) of the original MBR, using
\textbf{dd}. This includes only the boot loader part, without the partitions
information (which is also stored in the MBR). For the \textbf{grub2} disks,
the above method doesn't work, and the \textbf{grub-setup} utility needs to be
executed, like in \labelindexref{Listing}{lst:mbr-grub2} (the new system
partition is located on the \textbf{/dev/sdc} disk, and it is mounted in
\textbf{/mnt/new\_hdd/})

\lstset{caption=configuring grub2,label=lst:mbr-grub2}
\begin{lstlisting}
grub-setup -d /mnt/new_hdd/boot/grub /dev/sdc
\end{lstlisting}

After the new disks are configured, the \textbf{VMaster} is shut down and the
disks are detached from it. After the new disks are created and configured, the
newly created machine can be powered on to continue the generation process.

I tested the partitioning and cloning for the various supported systems:
Ubuntu, Debian, Windows XP, Windows 7.


\section{System Configuration (On Windows Running Guests)}
\label{sec:vmware-config}
I was responsible for implementing the Windows configuration and installation
modules. These modules are not VMware specific, but among the virtualization
solutions we used so far, VMware is the only one to support a Windows running
guest.

There is one problem with running commands on the guest system using
\textbf{vmrun}: the command to be executed cannot be too complex (cannot have
quoted arguments). Therefore, I chose to create a batch script, in which all
the desired commands are listed, and then the file is copied onto the guest and
executed there. Another advantage of the script is that there is no overhead
for executing each command through \textbf{vmrun} (\textbf{vmrun} is called
only one, to execute the script).

The \textbf{ConfigWindows} module has the methods described in
\labelindexref{Subsection}{sub-sec:configs}. To avoid creating a script file
for each configurations group (system, users, network, firewall), a string
containing the whole script is created, and each method writes the necessary
commands into it. In the end, the \textbf{applySettings} method is called,
which writes the resulted string into a file on the disk (\file{config.bat}),
and then copies the file onto the guest machine and executes it.

Although the documentation is not as detailed as the one for Linux, Windows has
some powerful command line utilities which can be used to configure the system,
like \textbf{net}, \textbf{netsh}, \textbf{wmic} etc. I gathered in a wiki
page\footnote{\url{http://ixlabs.cs.pub.ro/redmine/projects/vmgen/wiki/SystemConfig}}
the needed commands and arguments.

A sample generated \file{config.bat} file can be seen in
\labelindexref{Listing}{lst:config-bat}.

\lstset{caption=Sample generated config.bat,label=lst:config-bat}
\begin{lstlisting}
wmic COMPUTERSYSTEM where name="vmgen-pc" call rename "xp-gen"
net user Administrator pass2

net localgroup vmg /ADD
net localgroup julius /ADD
net user caesar alesia /ADD
net localgroup julius caesar /ADD
net user vv pass /ADD
net localgroup vmg vv /ADD

netsh interface ip set address name="Local Area Connection" static 192.168.1.2 255.255.255.0 192.168.1.1 1
netsh interface ip set dns name="Local Area Connection" static 192.168.1.254 primary

netsh firewall add portopening tcp 22 ssh
netsh firewall add portopening all 65000 myport
\end{lstlisting}

\textbf{wmic} is an utility used to display and change various system
configurations. More usage information can be obtained by running \cmd{wmic
/?}.

The \textbf{net} command is used to configure the groups, the users and to
control the system services. Only the users and groups features are used for
now, to create groups, create users, and add users to the specified groups.
Running \cmd{net help [section]} (where section can be blank and displays the
sections, or one of the section displayed without the extra argument).

For the networking and firewall configurations, the \textbf{netsh} utility is
used. It allows to see the current configuration and to change it. Usage
information can be displayed by running \cmd{netsh \ldots help}, where the
dots can be substituted by a chain of zero or more sections (e.g.:
\cmd{netsh help}, \cmd{netsh interface show help} etc.). For the firewall, there
can be defined both allowed ports and allowed applications (\project uses only
the ports feature).


\section{Application Installation (On Windows Running Guests)}
\label{sec:vmware-apps-win}
On Windows, unlike on Linux, there is no central repository of applications.
The supported applications are manually downloaded and stored in a folder on
the server. There is a dictionary file, in which are stored the details for
installing each application. If the administrator of \project wants to add
support for additional applications, he can download the setup kits and add the
corresponding entries to the dictionary.

Extensive documentation about installing application non-interactively on
Windows can be found at \cite{win-unattended}. There are 3 types of installers
supported by \project: \textbf{msi} installers, \textbf{simple executable}
installers and \textbf{archives}. There are different arguments to run the
installer in quiet mode (non-interactive). The \textbf{msi} installers use the
\textbf{msiexec} utility, written by Microsoft, and which has standardized
arguments. The executable installers, in general, have the \textbf{"/S"}
argument for quiet mode, but there are some exceptions.  The archives are
actually the application folder stored in a zip file. To install them, it is
sufficient to extract them in a default path (usually in
\file{C:{\textbackslash}Program Files}). To be able to extract them, the
machine needs to have an archiving application preinstalled (I preinstalled
7zip on the Windows machines, and added it to the system path, to be accessible
from every folder). A fragment of the dictionary used by
\textbf{InstallerWindows} is shown in
\labelindexref{Listing}{lst:installer-win-dict}. There are sample entries for
each type of installer, with or without additional arguments.

\lstset{caption=Program details dictionary,label=lst:installer-win-dict}
\begin{lstlisting}
programs = {
	"pidgin": {
		"type":"simple",
		"setup-file":"pidgin-setup.exe",
		"args":"/DS=1 /SMS=0 /L=1033 /S"
	},
	"vim":{
		"type":"simple",
		"setup-file":"gvim-setup.exe"
	},
	"python":{
		"type":"msi",
		"setup-file":"python-setup.msi"
	},
	"emacs":{
		"type":"script",
		"setup-file":"emacs.zip"
	}
	...
}
\end{lstlisting}

The installer module receives the list of applications to be installed and
generates a \file{setup.bat} file, in which are written the commands for
installing each application (see the previous section for why a script file is
created instead of executing the individual commands). The needed install kits
are added to an archive and copied, along with the \file{setup.bat} file to the
guest machine. The batch script also contains the command for the extraction of
the archive. The script is then executed on the guest machine, and the
applications are installed. After the installation is completed, the script
deletes all the kits and the initial archive from the guest machine. The script
is then deleted from the guest and from the local folder too. A sample
generated \file{setup.bat} file can be seen in
\labelindexref{Listing}{lst:setup-bat}.
 
\lstset{caption=Sample generated setup.bat,label=lst:setup-bat}
\begin{lstlisting}
7z.exe x -oe:\test\ "e:\test\setup.zip"
del "e:\test\setup.zip"
 e:\test\ThunderbirdSetup.exe -ms
 msiexec /qb /i e:\test\python-setup.msi 
del "e:\test\ThunderbirdSetup.exe"
del "e:\test\python-setup.msi"
\end{lstlisting}
