\chapter{Introduction}
\label{chapter:intro}

\section{Virtualization Technologies}
\label{sec:virt-tech}

\subsection{History}
\label{sub-sec:virt-history}
The virtualization is not a new technology. It was developed in 1960's by
IBM, for the mainframes. The problem with the mainframes back then was that
they were very expensive, and they were not used at their full capacity. The
virtualization was created to allow the partitioning of a large mainframe in
multiple virtual machines, in order to use the resources in a more efficient
way. When the x86 architecture came out, and servers on x86 started to be used,
virtualization didn't seem necessary, because the computers were cheap, and for
a while (about 10 years, between 1980's - 1990's) it was abandoned. Soon, the
need for virtualization began to rise once again, because the computers became
more and more powerful and they could do more than running a single OS. Unlike
the mainframes, which had hardware support for virtualization, the x86
architecture didn't. VMware developed in 1999 the first product which was able
to run multiple OS on the same x86 hardware. The problem with the x86
virtualization, was that there were a number of 17 CPU instructions that
couldn't be virtualized (the OS would crash or malfunction). VMware solved the
problem by trapping these instructions when they are generated, and replaces
them with virtualization-safe instructions. Later, more applications that
allowed virtualization of the x86 hardware, from other companies, appeared:
\textbf{VirtualPC}, by Connectix (later acquired by Microsoft), \textbf{Xen}
(an open-source hypervisor), \textbf{VirtualBox} by innoTek (later acquired by
Sun Microsystems, and now by Oracle), \textbf{lxc} and \textbf{OpenVZ}
(open-source OS-level virtualization solutions). For more details on the
virtualization technology history, \cite{practical-virt-solutions},
\cite{advanced-server-virt}, \cite{vmware-history} and
\cite{wiki-virt-timeline} can be consulted.

\subsection{Virtualization Types}
\label{sub-sec:virt-types}
Virtual machines are used for a wide range of purposes today. Depending on what
the virtual machine is needed for, a user may choose between various solutions.

\subsubsection{OS Level Virtualization}
\label{sub-sub-sec:virt-os-level}
OS level virtualization solutions allow for a more advanced chroot jail. The
virtual machines are called containers, virtual environments, or virtual
private servers. All the containers share the kernel of the host machine, so
the host must be running a Linux version, with support for the used technology.
The containers have the file system and processes isolated from the other
containers. The processes from the containers are scheduled by the host's
scheduler. The overhead of virtualization is minimum because the containers
don't run a full system inside them. This solution is used, for example, to run
multiple servers, isolated from each other, or to offer users separate running
environments, with minimum overhead. However, a problem exists with the kernel
being shared by all the containers: if one of the container runs some bad
kernel code and produce a kernel bug, all of the other containers and the host
machine are affected. Also, a different OS from the host's OS cannot be run in
a container.

Of the applications that offer OS level virtualization, I can mention
\textbf{OpenVZ} and \textbf{lxc}, which are currently supported in \project.

\subsubsection{Full System Virtualization}
\label{sub-sub-sec:virt-full}
The full system virtualization solutions allow running an entire OS (no need to
modify the OS to be able to run in a guest machine) inside the virtual machine.
The OS does not need to be the same as the host's (running, for example, a
Windows VM on a Linux system, or vice versa). Each VM uses virtual hardware,
for which there are special drivers. The overhead of running this type of VM is
higher than the overhead for an OS-level container. The scheduling for the
processes inside each VM is done by the VM's OS. The host's OS only schedules
the virtualization application's process. This solution is used to test a new
OS, without installing directly on the hardware, to run multiple OS at the same
time. It is also appropriate for kernel development and drivers programming:
when a programming error is done, the kernel bug affects only the VM, which can
be then restarted, or, better, reverted to a saved snapshot.

Examples of full system virtualization solutions are \textbf{VMware
Workstation}, \textbf{VirtualBox}, \textbf{VirtualPC}, \textbf{KVM} etc. Only
\textbf{VMware Workstation} is currently supported in \project.

\subsubsection{Hypervisors}
\label{sub-sub-sec:virt-hypervisors}
Both the previous virtualization solutions type, need an underlying OS running
on the physical machine. A hypervisor is a minimal piece of software that run
directly on the hardware, and offers the possibility to create and run multiple
virtual machines (guests) on top of it. The guests don't see the hardware
directly, they see a virtualization of it. The advantage is that multiple OS
can be run concurrently, without the overhead of an intermediate layer.
However, the performances are not the same as if the OS would run directly on
the hardware: the virtualized hardware needs special drivers, which cannot make
full use of the devices features. The OS can be an unmodified version, or a
modified version which is aware of the hypervisor used, and generates
instructions to be run on it (the hypervisor is transparent to the
applications, but not to the OS).

Some hypervisors are \textbf{Xen}, \textbf{VMware ESX}, \textbf{Hyper-V Server}
etc. \project doesn't currently support any hypervisor.

\section{Motivation}
\label{sec:motivation}
I will describe the motivation behind the implementation of \project, and which
are its goals.

As described above, the virtualization solutions are widely used. Creating a
virtual machine is a time consuming process, although it consists of relatively
simple operations, that can be automated. For one machine, it might not be a
big problem, but when a user wants to generate a couple of machines, some of
them having similar properties, the process becomes repetitive and it takes
more time. \project aims to automate the process of generating and configuring
a VM. The user only needs to make the request for the desired configuration(s),
and he will receive them when they are created, without any interaction during
the process.

When a user wants to create a network topology using some virtual machines (for
academic or other uses), he may need to install several machines with the same
configuration, but the network card configurations and maybe some services. It
would be easier to only change a few lines in a configuration file and request
a new machine instead of installing the machines from scratch.

Another use for \project would be for non-technical users or who don't know how
to configure a machine and the services on it, but they need the machine, in
order to interact with its services (e.g. a Web server). The user is able to
just specify the services to be installed, and it will be given the fully
configured machine.

In the following chapters, I will present how the application is implemented
and how the VM generation process works. More implementation details can be
obtained from the project's website (wiki and repository), at
\cite{vmgen-ixlabs}.
