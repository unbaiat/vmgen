\chapter{LXC}
\label{chapter:lxc}

In this chapter, the process of generating an \textbf{LXC} container will be
described in detail. The documentation I used can be found at \cite{lxc-teegra}
and \cite{lxc-zen}.

\section{Introduction to LXC Containers}
\label{sec:lxc-intro}
\textbf{LXC} is an OS level virtualization solution, which has native support
in the Linux kernel starting with version 2.6.29. It needs the \textbf{cgroup}
kernel functionality to work. It is relatively recent (from 2008), and the
documentation is pretty scarce. It took me a while to figure out the exact
steps needed to generate a working container. The advantage \textbf{lxc} has
over \textbf{OpenVZ} is the standard kernel integration. \textbf{OpenVZ} needs
to use a modified version of the kernel (not available for all distributions)
to be able to function.

A container consists of 3 parts: a configuration file, a mount points file
(fstab) and a directory containing the root file system. These will be detailed
in the next subsections.

\subsection{Host Machine Configuration}
\label{sub-sec:lxc-host-config}
To be able to run an \textbf{LXC} on a host machine, some configurations need
to be made on the host. The \textbf{cgroup} functionality must be configured in
the running kernel, then the \textbf{cgroup} file system must be mounted in
\textbf{/cgroup}. For the network adapters in the container to work, at least
one bridge interface must be configured on the host. A sample entry in the
\mbox{\textbf{/etc/network/interfaces}} for a Debian system is shown in
\labelindexref{Listing}{lst:lxc-bridge-cfg}. Finally, the \textbf{lxc}
utilities must be installed: \textbf{lxc-create}, \textbf{lxc-start},
\textbf{lxc-execute}, \textbf{lxc-stop} etc. These can be installed either
using the available package managers (\textbf{apt}, \textbf{yum}) or compiling
the sources from the project
site\footnote{\url{http://lxc.sourceforge.net/index.php/about/download/}}.

\lstset{caption=Bridge configuration,label=lst:lxc-bridge-cfg}
\begin{lstlisting}
auto br0
iface br0 inet dhcp
	bridge_ports eth0
	bridge_stp off
	bridge_maxwait 5
	post-up /usr/sbin/brctl setfd br0 0
\end{lstlisting}

\subsection{Configuration File}
\label{sub-sec:lxc-config-file}
The hardware of the container is described in a configuration file. The
configuration file is in plain text format, and consists of a list of key-value
associations, for setting various properties of the container. There are some
standard options present in the config file, which don't change across the
different container configurations. The complete list of options and more
details can be obtained from the manual page of lxc.conf (\textit{man
lxc.conf}). A sample config file is shown in
\labelindexref{Listing}{lst:lxc-config} (only the options that are relevant for
the user). The \textbf{lxc.network.*} options are used to specify the network
cards configurations. The location of the root file system and the
\textbf{fstab} file (both relative to the config file's location) are specified
by \textbf{lxc.rootfs} and \textbf{lxc.mount} respectively. The name of the
machine is set by the \textbf{lxc.utsname} option.

\lstset{caption=Sample config file,label=lst:lxc-config}
\begin{lstlisting}
lxc.utsname = TestMachine
lxc.rootfs = rootfs.TestMachine
lxc.mount = fstab.TestMachine
lxc.network.type = veth
lxc.network.link = br0
lxc.network.name = eth0
lxc.network.mtu = 1500
lxc.network.ipv4 = 172.16.30.160/24
lxc.network.flags = up
\end{lstlisting}


\subsection{Mount Points File}
\label{sub-sec:lxc-fstab-file}
The file containing the mount points (I will call it the \textbf{fstab} file) is
a plain text file, which contains the mount points to be used inside the
container. When the container is started, the will appear as the
\textbf{/etc/fstab} file. It can be defined inside the root file system, as
\textbf{\$rootfs/etc/fstab}, or it can be defined outside the file system, and
included in the container configuration file. A sample \textbf{fstab} file is
shown in \labelindexref{Listing}{lst:lxc-fstab}. The file is stored outside the
root file system, which is located in the \textbf{rootfs.TestMachine}
subdirectory.

\lstset{caption=Sample fstab file,label=lst:lxc-fstab}
\begin{lstlisting}
none rootfs.TestMachine/dev/pts devpts defaults 0 0
none rootfs.TestMachine/proc proc defaults 0 0
none rootfs.TestMachine/sys sysfs defaults 0 0
none rootfs.TestMachine/var/lock tmpfs defaults 0 0
none rootfs.TestMachine/var/run tmpfs defaults 0 0
/etc/resolv.conf rootfs.TestMachine/etc/resolv.conf none bind 0 0
\end{lstlisting}

\subsection{Root File System}
\label{sub-sec:lxc-rootfs}
The file system directory contains the OS file structure and all the container's
files. It is, in general, based on a minimal OS file structure, which is then
modified according to the specific requests. There are some utilities for
getting a minimum OS file system, like \textbf{debootstrap} for Debian systems,
and \textbf{febootstrap} for Fedora systems. These utilities download from the
Internet a specified version of the OS. To copy files inside the container, it
is sufficient to copy them in the corresponding relative path to the root
directory. The files of the container are also directly accessible, through
their relative paths. This is useful for editing the system configuration files
during the generation process.

I had some trouble using the \textbf{febootstrap} utility to download a Fedora
minimal OS. I found usage examples, but none of them worked, because it said
the arguments were invalid. I later found out that the latest version was 3
(which was also included in the system's repositories), which was completely
different from the second one. All the examples I found were for the second
version. I then downloaded the older version and the command worked as
expected. The only place where I found out that there are multiple completely
different versions of the program was the application's official
page\footnote{\url{http://people.redhat.com/~rjones/febootstrap/}}.

The container generation operation is distribution dependent. A Debian
container can be generated on a Debian running host machine, and the same goes
for a Fedora container. However, after a container is generated, it can be
deployed and used on any distribution, regardless of its type. \project
supports only 2 containers type: Debian and Fedora. A third option would be an
Ubuntu container, but I couldn't generate a working container. I considered
that Debian containers can be used instead of Ubuntu ones, because they have a
similar file structure and configuration files.

\section{The Setup}
\label{sec:lxc-setup}
Because of the distribution dependent generation process, and to avoid
cluttering the physical machine, the containers are created in a VMware virtual
machine (support machine). There is a machine running Debian and one running
Fedora. They are used to generate each container type. The machine that will be
used is selected based on the \textbf{os} field in the application
configuration file (more details in \labelindexref{Section}{sec:lxc-gen}). The
machines are configured as described in
\labelindexref{Subsection}{sub-sec:lxc-host-config}.  Both machines run a SSH
server, and they are configured to use public key authentication for the root
user. The use of public key authentication allows the same set of keys (a
private and a public key) to be used across the whole application, where
needed, and eliminates the need to remember the password of the root user in
each of the used machines. All the commands needed for the container creation
are executed through a SSH connection (even if not mentioned explicitly). 

\section{Container Generation}
\label{sec:lxc-gen}
The LXC archive, downloaded from the official site, contains some scripts for
generating container: \textbf{lxc-debian}, \textbf{lxc-fedora},
\textbf{lxc-ubuntu}. As I said earlier, the \textbf{lxc-ubuntu} didn't work.
The other 2 scripts seemed too complex, so I chose not to use them. Instead, I
created my own set of scripts (BASH scripts):
\textbf{my-lxc-debian.sh}\footnote{\url{http://blog.bodhizazen.net/linux/lxc-configure-debian-lenny-containers/}}
and
\textbf{my-lxc-fedora.sh}\footnote{\url{http://blog.bodhizazen.net/linux/lxc-configure-fedora-containers/}}.

A \textbf{CommanderLxc} and a \textbf{CommunicatorLxc} are used for the
container generation process. The \textbf{Communicator} uses direct file
manipulation on the container, and runs commands on the container using the
\textbf{lxc-execute} utility. These operations are executed on the VMware
virtual machine, through SSH.

The \textbf{CommanderLxc} module is the central module in the container
generation process. It reads the OS from the input configuration file, and it
calls the corresponding scripts, on the corresponding VMware machine to obtain
the container. The selections are made using a stored dictionary, indexed by
the OS identifier. The dictionary used to select the support machine and the script file is shown in \labelindexref{Listing}{lst:lxc-dict-vm}.


\lstset{language=Python,caption=Dictionary for selecting OS specific elements,label=lst:lxc-dict-vm}
\begin{lstlisting}
distro = {
"debian":{
	"vm":"/home/vmgen/vmware/Debian (lxc)/Debian (lxc).vmx",
	"hostname":"root@debian-lxc",
	"script":"my-lxc-debian.sh",
	"scripts-folder":"../scripts-lxc/debian/"},
"fedora":{
	"vm":"/home/vmgen/vmware/Fedora 64-bit/Fedora 64-bit.vmx",
	"hostname":"root@fedora-lxc",
	"script":"my-lxc-fedora.sh",
	"scripts-folder":"../scripts-lxc/fedora/"}
}
\end{lstlisting}

The support machine is powered on, then using the \textbf{Communicator}'s
methods, the scripts are copied from a local folder onto it and executed there.
The scripts for the 2 supported distributions are similar. First of all, the
specified version of the file system is downloaded, using \textbf{debootstrap}
or \textbf{febootstrap}. Because the download can be time consuming, an
alternative would be to have the file systems downloaded in a local folder, and
only copy them from there in the new location, instead of downloading them each
time. After the file system is downloaded, various system configuration files
are altered (using \textbf{sed}) or removed (some startup files). The root user
password is set to a default value, to preserve the modularity of the
application (it is the \textbf{Config} module's task to set the root password).
The container config file and fstab are then generated. A fragment of the code
used to generate the config file is shown in
\labelindexref{Listing}{lst:lxc-config-gen}. The used variables are initialized
before, with the appropriate values. The network settings are specified inside
the config file, by appending the needed options. After that, the container can be powered on, and the
\textbf{ConfigLinux} and the corresponding \textbf{Installer} module
(\textbf{InstallerApt} or \textbf{InstallerYum} are used to make the remaining
configurations and to install the specified applications and services inside
the container. An archive containing the container (config file, fstab file and
root file system) is then created, and returned to the user.

\lstset{language=Bash,caption=Generating the config file,label=lst:lxc-config-gen}
\begin{lstlisting}
cat << EOF > $config
lxc.utsname = $name
lxc.tty = 4
lxc.rootfs = rootfs.$name
lxc.mount = fstab.$name
lxc.cgroup.devices.deny = a
# /dev/null and zero
lxc.cgroup.devices.allow = c 1:3 rwm
lxc.cgroup.devices.allow = c 1:5 rwm
# consoles
lxc.cgroup.devices.allow = c 5:1 rwm
...
EOF
\end{lstlisting}
