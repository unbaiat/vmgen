\chapter{Further Development}
\label{chapter:future}
Now that the core application is implemented and tested, it can be extended, to
allow more functionalities and better user interaction.

\section{GUI}
\label{sec:future-gui}
The application is somewhat difficult to be used as it is right now. The user
needs to create from scratch a configuration file (or modify an existing one).
This process is error prone, some typos might occur. Also, the user needs to
know all the available options and possible values for them. 

To ease the user interaction with the application, a GUI would help.  The
addition of a graphical frontend will allow more users to try the application
and do not be discouraged by the complexity of the configuration file and
avoids the typing errors. The user will see a Web frontend, where he can select
the desired options. He is guided towards completing the whole file, and he is
presented the available options at each step. The frontend module then
generates a configuration file containing the options the user selected, and
then passes it to the main application. This modular approach prevents the need
to modify the existing application when adding a new frontend. In this way,
multiple frontends may be added, according to the \project administrator's
desire. The Web interface may look like the previously mentioned
\textbf{EasyVMX}\footnote{\url{http://www.easyvmx.com/}} online application.


\section{Additional Virtualization Solutions}
\label{sec:more-virt}
The supported virtualization solutions, so far, are \textbf{VMware},
\textbf{lxc} and \textbf{OpenVZ}. Support for other technologies can be added
in the future. The \textbf{Commander} and \textbf{Communicator} modules for
\textbf{VirtualBox} should be similar to the ones implemented for
\textbf{VMware}. Some research needs to be done and see how \textbf{KVM} and
\textbf{xen} virtual machines can be generated. After the needed operations are
found, the corresponding modules for these virtualization solutions can be
created. The modular architecture of the application allows for easy addition
of extra virtualization solutions, by implementing only the \textbf{Commander}
and \textbf{Config} corresponding modules. In the \textbf{main} method, the
selection of the module to be instantiated must be also modified, to include
the new modules.


\section{Additional Options In The Configuration File}
\label{sec:more-opts}
Support for more installable applications can be added by adding the
corresponding entries in the dictionaries of the \textbf{Installer} modules. In
the future, there should be the possibility to provide additional parameters
for the installers (like installation path, shortcuts created, additional
settings after installation). This can be done by providing a script file
containing the exact commands to be run by the installer process. This has the
disadvantage that the user needs to know exactly the commands needed. It also
presents a security risk, by allowing the user to enter code that will be
executed directly, but fortunately, it is only executed on the guest virtual
machine, and the potential damage is limited to only the guest. This feature is
essential for configurable services, like a Web server, a DNS server, a DHCP
server etc. These need additional settings to be made in order to be useful.

Also, the options available for specifying the hardware components are fairly
basic, it should be possible to allow the user to specify more advanced
settings, if he wishes to. Only a limited number of OS are supported at the
moment. Support for more OS should be added in the future, by preparing the
needed resources (preinstalled base installation disks for \textbf{VMware},
support machine for \textbf{lxc} and \textbf{OpenVZ} etc).
