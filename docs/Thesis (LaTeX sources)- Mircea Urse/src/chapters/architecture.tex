\chapter{Application Architecture}
\label{chapter:architecture}

\section{Application Deployment And User Interaction}
\label{sec:app-deploy}
\project is designed to run as a service on a configured remote server. After
it is started, it begins to listen for user requests. At the moment, it cannot
process multiple requests simultaneously, they are processed sequentially. The
reason why it has to run on a configured server is that, in the process of
generating the requested machines, it needs to use some preinstalled virtual
machines, and some installation kits for the applications to be installed. The
preinstalled virtual machines cannot be accessed concurrently, and there needs
to be implemented a synchronization mechanism to enforce mutual exclusion. At
the moment, a request gains exclusive access during the whole process. In the
future, it should be changed so that a requests needs to gain exclusive access
only on the critical sections of the process.

The application receives a configuration file as input and generates an archive
containing the generated virtual machine after it has finished the generating
process. At the moment, the user can write himself the whole configuration
file, or he can use our \textbf{vmgCtl} utility to alter an existing
configuration file; an example of usage is shown in
\labelindexref{Listing}{lst:vmgctl-exp}. There is no Web interface for now, but
it will be implemented in the future, as it will be presented in
\labelindexref{Section}{sec:future-gui}. No matter what frontend is added, it
generates a configuration file based on the user input, and passes the
generated file to the main application, which is not aware of the frontend.

\lstset{caption=vmgCtl usage example,label=lst:vmgctl-exp}
\begin{lstlisting}
./vmgCtl.py config-debian.conf hardware.num_cpu=2
\end{lstlisting}

\subsection{The Programming Language}
\label{sub-sec:prog-lang}
The application code is written in \textbf{Python}. The reason we chose Python
was because it is very easy to write code in it, and the code is more readable
than C code, for example. Python has a large collection of features that are
already implemented that we could use (command execution, file manipulation,
archive file manipulation, list comprehension etc.). We could focus on
implementing the application features rather than re-implementing the small
pieces of code for different operations. Also, the written code is
cross-platform, so it could be run either on a Linux machine or on a Windows
one. The application doesn't need critical performance to run, so the slower
execution of Python code is not a problem.

\subsection{Configuration File}
\label{sub-sec:config-file}
The configuration file is a plain text file, with the extension \textbf{.conf},
in a modified
\textbf{INI}\footnote{\url{http://en.wikipedia.org/wiki/INI\_file}} format.
Besides the sections, the configuration file has also nested subsection,
defined by using multiple square braces, according to the nesting level. An
alternative format would be
\textbf{XML}\footnote{\url{http://en.wikipedia.org/wiki/Xml}}, but the file
would get even larger than it is and would not be very easily readable and
editable by the user. A detailed description of each section and examples
follow. The currently supported options are available on the project
wiki\footnote{\url{http://ixlabs.cs.pub.ro/redmine/projects/vmgen/wiki/SupportedOptions}}.

\subsubsection{The hardware section}
\label{sub-sub-sec:hardware-sec}
In the \textbf{hardware} section, the user describes the hardware
characteristics of the machine. The available options are different between the
used virtualization solutions, but most of them are common among all the
solutions. The user can specify the ID of the virtual machine (name or number),
the operating system, the number of processors, the available physical memory,
the number of hard disks and their parameters, the partitioning details for
each of them, the CD drives, the network adapters. 

\lstset{caption=sample hardware section,label=lst:hardware-section}
\begin{lstlisting}
[hardware]
	vm_id = TestMachine
	os = winxppro
	num_cpu = 2
	ram = 512
	[[hdds]]
		[[[hdd0]]]
		size = 8GB
		type = ide
		scsi_index = 0
		pos = 0:0
		name = hdd.vmdk
		[[[[partitions]]]]
			[[[[[partition0]]]]]
			type = primary
			fs = ntfs
			size = 5000
			...
		...
	[[cd_drives]]
		[[[cd_drive0]]]
		pos = 1:0
		path = /path/to/iso
		connected = 1
		...
	[[eths]]
		[[[eth0]]]
			type = nat
			connected = 1
			hw_addr = 01:00:de:ad:be:ef
		...
\end{lstlisting}

A very important field is \textbf{os}. It specifies the OS which will run in
the virtual machine, along with its architecture (ubuntu-64, fedora, winxxppro,
windows7-64 etc). The identifiers are the ones used in VMware machine
description files. All the OS dependent selections (the OS installation,
\textbf{Config} and \textbf{Installer} modules instantiation, system disk
cloning) will be made based on this field. The application has dictionaries
indexed by the OS identifier, whose keys are the various alternatives. For more
details about the subject, see \labelindexref{Section}{sec:vmware-os}.


\subsubsection{The config section}
\label{sub-sub-sec:config-sec}
The \textbf{config} section is used to specify some system parameters, like the
password for the root user (or Administrator, on Windows), the hostname,
additional repositories (for \textbf{apt} or \textbf{yum}) etc. These are
generic settings that do not fall in the other categories (network, users,
services, applications).

\lstset{caption=sample config section (Linux guest),label=lst:config-section}
\begin{lstlisting}
[config]
	root_passwd = pass
	hostname = trudy
	bash_completion = 1
	[[repos]]
		repo0 = deb http://http.us.debian.org/debian stable main contrib non-free
	...
\end{lstlisting}

\subsubsection{The users section}
The \textbf{config} contains the information about the groups and the users in
the new system. All the groups from the \textbf{groups} subsection are created.
The users from the \textbf{users} subsection are created with the specified
name, password and home directory, and they are added to the specified groups
afterwards.

\label{sub-sub-sec:users-sec}
\lstset{caption=sample users section (Windows guest),label=lst:users-section}
\begin{lstlisting}
[users]
	[[groups]]
		group0 = julius
		group1 = vmg
		...
	[[users]]
		[[[user0]]]
			name = caesar
			passwd = alesia
			groups = julius
			home_dir = C:\Users\caesar
		[[[user1]]]
			name = vv
			passwd = pass
			groups = vmg
			home_dir = C:\Users2\vv
		...
\end{lstlisting}


\subsubsection{The network section}
\label{sub-sub-sec:network-sec}
The \textbf{network} section is a larger one, and it groups all the
configurations available for networks: network applications, network adapter
configurations, firewall rules.

The network adapter configurations are a little different between Linux and
Windows guests. There are some common properties, like the type of addresses
(static or dynamic), the IP address, the network mask. These are specified for
each adapter. The difference is at the DNS address and the gateway. On Linux,
we have global DNS servers (in \file{/etc/resolv.conf}) and a single default
gateway, with multiple additional routes. On the other hand, in Windows, each
network adapter has its own gateway and DNS addresses. Therefore, for Linux
guests, there are global properties for the DNS and gateway, and for Windows,
the properties are for each adapter.

The firewall can also be configured. The user can specify a list of ports to be
opened, along with the protocol and a short description for each port. If the
user wants to specify more complex firewall rules, he can specify in the
\textbf{firewall-rules} subsection the exact commands (\cmd{iptables} on
Linux or \cmd{netsh firewall} on Windows) to be run on the machine. It is
not a very good practice to run the exact commands provided by the user,
because he can try to run some malicious code, but these commands are executed
only inside the virtual machine, and only this will be affected, not the
physical machine running the service.

\lstset{caption=sample network section (Windows guest),label=lst:network-section}
\begin{lstlisting}
[network]
	[[eths]]
		[[[eth0]]]
			type = static
			address = 192.168.1.2
			network = 255.255.255.0
			gateway = 192.168.1.1
			dns = 192.168.1.254
		[[[eth1]]]
			type = dhcp
		...
	[[open_ports]]
		[[[port0]]]
			proto = tcp
			port = 22
			description = ssh
		[[[port1]]]
			proto = all
			port = 65000
			description = myport
		...
\end{lstlisting}

\subsubsection{The devel section}
\label{sub-sub-sec:devel-sec}
The \textbf{devel} section specifies the development tools to be installed:
development environments, compilers, profilers, debuggers, libraries etc.

\lstset{caption=sample devel section (Linux guest),label=lst:devel-section}
\begin{lstlisting}
[devel]
	vim = 1
	emacs = 1
	eclipse = 1
	build-utils = 1
	kernel-devel = 1
	valgrind = 1
	python = 1
	php = 1
	tcl = 1
	...
\end{lstlisting}


\subsubsection{The services section}
\label{sub-sub-sec:services-sec}
In the \textbf{services} section are specified the services to be installed on
the machine. The main options here are servers, like Web, mail, DNS, DHCP, FTP
etc. There aren't many parameters to configure for them at the moment, but it
might be a good feature to work on in the future. One of the reasons why this
feature is not implemented yet is that there has to be a way to specify these
options in the config file and we didn't find a suitable way to do that. 

\lstset{caption=sample services section,label=lst:services-section}
\begin{lstlisting}
[services]
	httpd = 1
	dns-server = 1
	dhcp-server = 1
	ftp-server = 1
	sshd = 1
	svn = 1
	git = 1
	...
\end{lstlisting}


\subsubsection{The gui section}
\label{sub-sub-sec:gui-sec}
The \textbf{gui} section lists the applications to be installed, which need a
graphical environment to run (non-CLI).

\lstset{caption=sample gui section,label=lst:gui-section}
\begin{lstlisting}
[gui]
	mozilla-firefox = 1
	google-chrome = 1
	mozilla-thunderbird = 1
	wireshark = 1
	...
\end{lstlisting}

\section{Modules Description}
\label{sec:mod-desc}

\project consists of several modules. \labelindexref{Figure}{img:vmgen-uml}
gives an overview over the general architecture of the application. The modules
and the relationship between them will be described in detail below.  The
application is executed in command line, and needs to receive as arguments the
virtualization solution (vmware, lxc, openvz so far) and the configuration
file. A parser module reads the file and passes the retrieved data to the main
generation process. The generation process consists of various stages. The
main stages are \textbf{hardware generation}, \textbf{system configuration} and
\textbf{application installation}. The operations executed in these stages can
be applied to more than one configuration. For example, in the \textbf{hardware
generation} stage, the operations are almost OS independent, and are different
only across the virtualization solutions. In the \textbf{system configuration}
stage, the operations are distribution and virtualization solution independent,
and are common for each OS family (Linux and Windows).  The \textbf{application
installation} stage is common for each installer type (\textbf{apt},
\textbf{yum}, \textbf{source installation} for Linux, or the installation kits
on Windows). Also, a very important component, used in all the other one is the
one responsible for the communication with the virtual machine. The operations
needed to communicate with a given machine are dependent only on the machine
type (virtualization solution used).

\fig[scale=0.35]{src/img/vmgen-uml.pdf}{img:vmgen-uml}{\project architecture}

Given these observations, we tried to design the application to be modular. One
of our goals was to not have duplicated code, so we grouped the operations that
could be used in more than one place in a separate module and use that module
instead. This provides easier code maintenance. The other goal was to design an
extensible application. If later on someone wants to add support for a new
feature (a new virtualization solution, new applications to be installed etc.),
it is sufficient to create a new corresponding module, link it in the main
application, and use the already implemented modules where needed. 

In \labelindexref{Figure}{img:app-flow} is presented the virtual machine
generation process, from input to output. I will give a short overview over the
functionality of the modules. Each module will be detailed in the next
subsections. The user provides the configuration file for the machine he wishes
to get. The \textbf{Parser} loads the file into memory and passes it to a
\textbf{Commander}. The \textbf{Commander} generates the hardware of the
machine and configures the OS, then it instantiates a \textbf{Config} and a
\textbf{Installer} module. The \textbf{Config} makes the necessary system
settings (users, network etc.). The \textbf{Installer} installs the requested
applications and services. Then, an archive containing the virtual machine is
created and made available to the user to download it. 

\fig[scale=0.35]{src/img/application-flow.pdf}{img:app-flow}{VM generation process}

\subsection{Configuration File Parser}
\label{sub-sec:parser}
The parser is a simple module, which uses the
\textbf{ConfigObj}\footnote{\url{http://wiki.python.org/moin/ConfigObj}}
module, from the Python library to parse the configuration file and store it in
a \textbf{vmgStruct} structure. After the structure is created and populated
into memory, it is serialized in a dump file, using the
\textbf{pickle}\footnote{\url{http://docs.python.org/library/pickle.html}}
module (also from the Python library), in orderd to be de-serialized by the
next component in the chain. The serialization offers a decoupling between the
parser and the next stages of the application.

\subsection{Commander Modules}
\label{sub-sec:commanders}
The main component of \project is the \textbf{Commander}. The commander is
instantiated at the beginning of the generation process and it basically
controls the other modules. It receives the previously created dump file, and
recreates the \textbf{vmgStruct} in memory. A commander uses the information
from the \textbf{hardware} section in the config file
(\labelindexref{Subsection}{sub-sub-sec:hardware-sec}). 

Because the steps needed to configure the virtual machine depend on what
virtualization solution is used, there must be a commander for each supported
virtualization solution. To keep the structure modular, an abstract class,
\textbf{CommanderBase}, is used. This base class splits the generation process
into smaller steps, and executes them in the correct order. For each step,
there is an abstract method, which must be implemented by each of the concrete
commanders. The abstract methods can be seen in
\labelindexref{Listing}{lst:commander-methods}.

\lstset{language=Python,caption=CommanderBase methods,label=lst:commander-methods}
\begin{lstlisting}
class CommanderBase:
	...
	def startVM(self):
	def shutdownVM(self):
	def connectToVM(self):
	def disconnectFromVM(self):
	def setupHardware(self):
	def setupPartitions(self):
	def setupOperatingSystem(self):
	def setupServices(self):
	def setupDeveloperTools(self):
	def setupGuiTools(self):
	...
\end{lstlisting}

Only the \textbf{setupVM} method is called on a specific Commander instance.
This method will, in turn, call the needed operations. The sequence of
operations needed to configure a new virtual machine can be seen in
\labelindexref{Listing}{lst:commander-seq}. The commander creates the hardware,
partitions the disks, and setups the OS on the new machine, then it starts it
and connects to it. The system configuration is made by instantiating the
corresponding \textbf{Config} module (see
\labelindexref{Subsection}{sub-sec:configs}) for the specified OS and calling
its main method. The final step before turning the machine off is installing
the applications, by category: services, developer tools, graphical
applications. The installations are done by instantiating the corresponding
\textbf{Installer} module (see \labelindexref{Subsection}{sub-sec:installers}).
Each of these operations is implemented differently by the commanders.

\lstset{language=Python,caption=Commander sequence of steps,label=lst:commander-seq}
\begin{lstlisting}
def setupVM(self):
	self.setupHardware()
	self.setupPartitions()
	self.setupOperatingSystem()

	self.startVM()
	self.connectToVM()

	self.config = self.getConfigInstance()
	self.config.setupConfig()
	self.root_passwd = self.config.getNewRootPasswd()

	self.installer = self.getInstallerInstance()
	self.setupServices()
	self.setupDeveloperTools()
	self.setupGuiTools()

	self.disconnectFromVM()

	self.shutdownVM()
\end{lstlisting}

For each virtualization solution, a new commander is created by deriving the
\textbf{CommanderBase} class and implementing the corresponding operations:
\textbf{CommanderLxc}, \textbf{CommanderOpenvz}, \textbf{CommanderVmware} etc.
An advantage of this organization is that a user who wants to add support for a
different virtualization solution needs to define a new derived
\textbf{Commander}, along with a corresponding \textbf{Communicator} (see
\labelindexref{Subsection}{sub-sec:communicators}) and add the particular
implementations for its operations. The only modification needed to be made to
the main application is at the top level, where the commanders are
instantiated, to add the instantiation code for the new commanded.

The relationship between the classes can be seen in \labelindexref{Figure}{img:commanders}.

\fig[scale=0.35]{src/img/commanders.pdf}{img:commanders}{Commander modules}

For now, only 3 commanders are implemented, some of which will be presented in
detail later: CommanderLxc (\labelindexref{Chapter}{chapter:lxc}),
CommanderVmware (\labelindexref{Chapter}{chapter:vmware}) and CommanderOpenvz.
In the future, more Commanders can easily be added to the application (more
details can be found in \labelindexref{Section}{sec:more-virt}).

\subsection{Configuration Modules}
\label{sub-sec:configs}
A \textbf{Config} module is responsible for configuring the OS on the virtual
machine. The options used by the \textbf{Config} modules are placed in the
\textbf{config} section in the config file
(\labelindexref{Subsection}{sub-sub-sec:config-sec}).

There is an abstract class, \textbf{ConfigBase}, which has abstract methods for
each configuration group: \textbf{system configurations}, \textbf{users and
groups}, \textbf{network}, \textbf{firewall} and a method for applying the
settings generated by the previous ones. These methods are shown in
\labelindexref{Listing}{lst:config-methods}.

\lstset{language=Python,caption=ConfigBase methods,label=lst:config-methods}
\begin{lstlisting}
class ConfigBase:
	def setupSystem(self):
	def setupGroups(self):
	def setupUsers(self):
	def setupNetwork(self):
	def setupFirewall(self):
	def applySettings(self):
	...
\end{lstlisting}

There is also a method, \textbf{setupConfig}, which calls each abstract method,
to execute each configuration step. The sequence of steps is shown in
\labelindexref{Listing}{lst:config-seq}. The \textbf{getRootPasswd} method is
used to return the stored password set for the privileged user during the
configuration process, for later use. 

\lstset{language=Python,caption=Config sequence of steps,label=lst:config-seq}
\begin{lstlisting}
def setupConfig(self):
	self.setupSystem()
	self.setupGroups()
	self.setupUsers()
	self.setupNetwork()
	self.setupFirewall()

	self.applySettings()
\end{lstlisting}

A new \textbf{Config} module is created for each OS family:
\textbf{ConfigLinux} and \textbf{ConfigWindows} by deriving
\textbf{ConfigBase}.  Each concrete \textbf{Config} class must implement the
configuration abstract methods, according to the OS family it implements.  The
relationship between the classes can be seen in \labelindexref{Figure}{img:configs}.

\fig[scale=0.35]{src/img/configs.pdf}{img:configs}{Config modules}

A \textbf{Config} module is instantiated by the \textbf{Commander}, based on
the operating system it read from the configuration file (in the hardware
section). The \textbf{Commander} has a dictionary, whose keys are the possible
values of the OS, and its values are the various concrete \textbf{Config}
classes. After the \textbf{Commander} module instantiates the corresponding
\textbf{Config} module, it needs to call only the \textbf{setupConfig} method
to make all the necessary configurations. It then needs to retrieve the new
password for the privileged user by calling the \textbf{getRootPasswd} method.


\subsection{Installer Modules}
\label{sub-sec:installers}

The \textbf{Installer} module is used for installing applications inside the
generated virtual machine. The main application does not need to know how the
programs are installed, it needs to only send the install command for a
specific program. The used tools and the needed operations are encapsulated
inside an \textbf{Installer} module. The installing process of an application
is not OS dependent, but rather depends on the installation tools provided by
the OS. Some of these tools are \textbf{apt} for Debian based systems,
\textbf{yum} for RedHat based systems, the installation from \textbf{sources},
for all the Linux distributions, individual \textbf{executable} kits, for
Windows etc.

The \textbf{Installer} module is used to install the applications specified in
various sections of the config file, like \textbf{devel}
(\labelindexref{Subsection}{sub-sub-sec:devel-sec}), \textbf{services}
(\labelindexref{Subsection}{sub-sub-sec:services-sec}) and \textbf{gui}
(\labelindexref{Subsection}{sub-sub-sec:gui-sec}).

The only method needed for an \textbf{Installer} module is one that receives a
list of program names and installs them all. This method is provided by the
abstract class \textbf{InstallerBase}. The definition of the class is shown in
\labelindexref{Listing}{lst:installer-methods}.

\lstset{language=Python,caption=Installer methods,label=lst:installer-methods}
\begin{lstlisting}
class InstallerBase:
	def install(self, programList):
\end{lstlisting}

For each installation tool supported, a
derived class from \textbf{InstallerBase} is created, like
\textbf{InstallerApt}, \textbf{InstallerYum}, \textbf{InstallerWindows},
\textbf{InstallerSource}. So far, only the first 3 installer types are
supported. The relationship between the classes can be seen in
\labelindexref{Figure}{img:installers}.

\fig[scale=0.35]{src/img/installers.pdf}{img:installers}{Installer modules}

Internally an \textbf{Installer} module stores a dictionary, for associating a
generic name for an application (OS and tool independent) with the real name
and the parameters needed to install the application, using the module's
specific installation tool. To add new applications, it is sufficient to add
the corresponding entries in the dictionary, and they will be installable after
that, without modifying anything else. The applications are installed with the
default parameters, in the default path (usually in
\textbf{C:{\textbackslash}Program Files}). We could not found a suitable way to
specify additional installation options for programs (like destination path,
shortcuts created etc.), without making the configuration file too complex. A
solution would be to specify the desired parameters for a program installation
in a separate file, and in the main configuration file to provide the link to
that file.


\subsection{Communicator Modules}
\label{sub-sec:communicators}

One of the most important modules is the \textbf{Communicator} module. It
allows the rest of the components to communicate with the virtual machine
(\labelindexref{Figure}{img:vm-comm}).  The rest of the modules don't need to
know which machine are they sending commands to. They have a reference to a
\textbf{Communicator} object, corresponding to the virtual machine, and they
use the interface of the communicator to interact, indirectly, with the
machine. Without the use of \textbf{Communicator} modules, the
\textbf{Installer} and \textbf{Config} modules, which are independent of the
virtualization solution used would have to decide by themselves which machine
they are communicating with, and this code would be duplicated across their
various implementations.

\fig[scale=0.35]{src/img/vm-communication.pdf}{img:vm-comm}{Communication with the VM}

A communicator provides methods to run commands in the machine, to copy files
from the local (physical machine) to the guest (VM), to remove files inside the
guest. An abstract class, \textbf{CommunicatorBase} provides this interface.
Its definition is presented in \labelindexref{Listing}{lst:comm-methods}.

\lstset{language=Python,caption=Communicator methods,label=lst:comm-methods}
\begin{lstlisting}
class CommunicatorBase:
	def runCommand(self, cmd):
	def copyFileToVM(self, localPath, remotePath):
	def deleteFileInGuest(self, remotePath):
\end{lstlisting}

A \textbf{Communicator} for each virtualization solution is created, by
deriving \textbf{CommunicatorBase}. So far, we have created the derived classes
for the supported virtualization solutions: \textbf{CommunicatorLxc},
\textbf{CommunicatorOpenvz}, \textbf{CommunicatorVmware}. The relationship
between the classes can be seen in \labelindexref{Figure}{img:communicators}.

\fig[scale=0.35]{src/img/communicators.pdf}{img:communicators}{Communicator
modules}

The \textbf{CommunicatorVmware} module uses VMware's \textbf{vmrun} utility (at
\cite{vmrun-man} can be found the official manual) to directly run
applications, copy files inside the virtual machine, and delete files from it.

The \textbf{CommunicatorLxc} and \textbf{CommunicatorOpenvz} do not interact
directly with the \textbf{lxc} and \textbf{OpenVz} containers. As will be
presented in \labelindexref{Chapter}{chapter:lxc}, the containers are not
generated on the physical machine, but in a VMware virtual machine. So, these 2
communicators connect to the VMware machine through SSH, using public key
authentication, and the execute the container specific commands to run programs
inside the container. To copy files to the container, it is sufficient to
simply copy through SSH (using \textbf{scp}) the files into the directory where
the file system of the container resides. To delete files from the container,
the steps are similar to the ones for copying, but instead of copying a file,
the \textbf{rm} command is run through SSH, using the path of the container's
file system.
